\documentclass{practice}

\usepackage{enumitem}
\usepackage{tcolorbox}

\usepackage{tikz}
\usetikzlibrary{positioning,calc}

\newcommand*{\ENC}{\mathsf{Enc}}

\usepackage{listings}

\lstset{
basicstyle=\small\ttfamily,
columns=flexible,
breaklines=true
}

\begin{document}
\begin{center}
Homework 01 --- Deadline: 01.03.24 23:59 EET
\end{center}

\textbf{Technical instructions}

\begin{itemize}
  \item Theory part:
  \begin{itemize}
    \item Submit on Moodle as a PDF called \texttt{studentid-01.pdf} where `studentid' is your 6 letter student ID/email username (`takraa' for me).
    \item Add to your report roughly how long it took you to solve the complete homework (theory + code).
    \item Include your Discord username in the PDF if you want me to give you feedback on Discord instead of Moodle (make sure to whitelist the server for DMs).
  \end{itemize}

  \item Practical/code part:
  \begin{itemize}
    \item Submit your code part files as a ZIP bundle on Moodle.
    \item I must be able to run your code using the CLI parameters defined in the templates.
    \item Your code must run with Python 3.11.
    \item You may use any standard module, but only the following 3\textsuperscript{rd} party libraries:
    \begin{itemize}
      \item \href{https://pypi.org/project/pycryptodome/}{\texttt{pycryptodome}}, \href{https://pypi.org/project/PyNaCl/}{\texttt{PyNaCl}}, \href{https://pypi.org/project/cryptography/}{\texttt{cryptography}}
    \end{itemize}

    Confirm with me beforehand if you want to use anything else.
  \end{itemize}
\end{itemize}

\textbf{Organisational details}

\begin{itemize}
  \item You may request deadline extensions until 26.02.24 23:59 EET.
  \item If you cannot solve the code tasks, explain your issues instead \& show that you tried.
  \item You may give hints to each-other (but no answer/code sharing) and you may ask me questions in private or publicly (discord server/Moodle forums).
  \item The only way to fail the homework is by:
  \begin{itemize}
    \item not submitting it on time
    \item submitting an incomplete homework \emph{without} explanations
    \item plagiarising/copying (in which case I will also report you to the faculty)
  \end{itemize}
  \item If you fail the homework, you fail the course: you cannot attend the exam unless you have passed all homework assignments.
  \item If I mark your homework as failed you can appeal my decision:
  \begin{itemize}
    \item You must do so within 3 days of receiving your grade (grade day + 3).
    \item We will have a homework defence/impromptu oral exam.
    \item If you fail the defence, the fail is permanent.
  \end{itemize}
  \item I grade homework ASAP:
  \begin{itemize}
    \item If you submit before the deadline, you may get feedback on issues that you can still fix before the deadline.
    \item The closer to the deadline you submit, the less likely I can review it before the deadline.
  \end{itemize}
\end{itemize}

\newpage

\begin{center}
  Theory tasks
\end{center}

\textbf{Task 1 --- A little about you}

What role are you currently working as/what would you like to work as?
Is or do you anticipate cryptography to be somehow related to your work?
Was there any specific reason (other than the 6 ECTS) that made you pick this course/what do you hope to gain from it?

\textit{Just a few sentences (3--4) is fine.}

\textbf{Task 2 --- Knowing your system}

What operating system do you use? Give three examples of disk-encryption software supported on your OS (include links).

Do you use full disk encryption? If not, would you start using it?

\textbf{Task 3 --- Knowing your tools}

What programming language do you use most/are you most comfortable in?
For this language, list the following:
\begin{enumerate}
  \item Standard library/namespace/module (if any) for getting cryptographic randomness.
  \item Standard library/namespace/module (if any) for encryption.
  \item Three well regarded 3\textsuperscript{rd} party libraries for cryptographic stuff (any kind).
\end{enumerate}

Include your sources!

\textbf{Task 4 --- Cryptographer's Rosetta}

If your mother tongue is not English, try to find a frequency diagram for your language (as we did for the shift cipher).

If your mother tongue is English, or you cannot find a diagram for your mother tongue, find a frequency diagram for two (or three) letter pairs (e.g. `at', `to', `no', \dots).

Include the source!

\textbf{Task 5 --- Knowing your processor}

Benchmark how long it takes for your computer to iterate over the 32-bit space (practice 1). How long would it take for your computer to cover the 128-bit space based on this info? Show your calculations.

\textbf{Task 6 --- A pledge}

`I understand that I should not implement my own cryptography for any production system, i.e. for any purpose other than a personal learning challenge.
I understand also that any implementations done during this course are only for learning purposes, and are not for real world use. I understand that due diligence is crucial.'

\textit{Answer with yes/no.}

\newpage

\begin{center}
  Practical tasks
\end{center}

\textbf{Task 1 --- Cryptohack}

Create an account on \url{https://cryptohack.org/courses/}, and try to finish at least the 10 courses of the `Introduction to Cryptohack'.
If you cannot finish the 10 courses, write in your report what tasks you were stuck with and what did you try.

Include a screenshot of your module completion and your username in the PDF report.

I do recommend that you to attempt the `Symmetric Cryptography' module as well, but that is not a mandatory part of the homework.

\textbf{Task 2 --- AES CBC from ECB}

Homework template file: \texttt{aes\_ecb\_cbc\_template.py}

The task is to use a library implementation of AES (you need not implement AES yourself) in ECB mode, and build the CBC mode out of it.

For example, if you use PyCryptodome, you should instantiate AES with
\begin{center}
  \texttt{aes = AES.new(key, AES.MODE\_ECB)}
\end{center}

Your program should take in the key and message from the CLI, and have an optional parameter for the IV (see template).
The key, message, and IV are provided as hex-encoded on the CLI.

\begin{tcolorbox}[title=Note]
  In practice, providing secrets (e.g. keys, passwords, etc) directly on the CLI as an argument to a program is a terrible idea, since they typically get logged into your terminal history!
\end{tcolorbox}

Additional program requirements:
\begin{itemize}
  \item If the IV is provided with the CLI, it must use the IV. Else, it must generate a \emph{random} IV.
  \item It must work with any key-size supported by AES.
  \item It must encrypt the provided message and print out the hex-encoded ciphertext (without the IV).
  \item It must support multi-block messages.
  \item It should not crash if a dubious key/IV is provided: handle your errors!
\end{itemize}

Example call (with NIST AES MTT data):%
\begin{verbatim}
python3 aes_ecb_cbc.py \ 
  4278b840fb44aaa757c1bf04acbe1a3e \
  3c888bbbb1a8eb9f3e9b87acaad986c466e2f7071c83083b8a557971918850e5 \
  --iv 57f02a5c5339daeb0a2908a06ac6393f
\end{verbatim}

Do not remove/modify the \texttt{print} or \texttt{assert} statements in the template.

%Some hints: https://www.pycryptodome.org/src/cipher/aes

%https://www.pycryptodome.org/src/util/util#Crypto.Util.Padding.pad
%https://www.pycryptodome.org/src/util/util#crypto-util-strxor-module

%AES.block_size
%AES.key_size[0]

\textbf{Task 3 --- ChaCha20 encryption/decryption}

Homework template file: \texttt{chacha20\_template.py}

Encrypt the file \texttt{plaintext.txt} $5$ times using ChaCha20 with the same key and output the ciphertext bytes into files \texttt{ct-1.bin}, \texttt{ct-2.bin}, \dots.

\begin{tcolorbox}[title=Note]
  Make sure that you write the raw ciphertext bytes to the file, and not the bytes of a hex or base64 encoded ciphertext.
\end{tcolorbox}

Write the key into a file called \texttt{key.txt} as a hex string, e.g.
\begin{tcolorbox}
  \texttt{key.txt}
  \tcblower
  \texttt{4278b840fb44aa...}
\end{tcolorbox}

Write the data needed for the decryption of each file, one hex string per line, into a file named \texttt{decryption.txt}.
Remember to clear the file at the beginning of each encryption cycle, i.e. there should only be 5 lines in the file.

Test your program!
\begin{itemize}
  \item Your program should be able to decrypt all ciphertexts!
  \item Your program should not crash!
\end{itemize}

Include the generated files \texttt{ct-\{1-5\}.bin}, \texttt{key.txt}, \texttt{decryption.txt} into your submission ZIP.

Example call:%
\begin{verbatim}
python3 chacha20.py encrypt 4278b840fb44aa... message.txt ./
\end{verbatim}

\end{document}
