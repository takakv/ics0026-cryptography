\usetheme[numbering=fraction]{metropolis}

\metroset{block=fill}

\usepackage[dvipsnames]{xcolor}

\usepackage{booktabs}

\usepackage{fancyvrb}
\fvset{listparameters=\setlength{\topsep}{0pt}\setlength{\partopsep}{0pt}}

\usepackage{csquotes}
\setquotestyle{british}

\usepackage{graphicx}
\graphicspath{ {../../images/} }

\usepackage{pgfplots}
\usetikzlibrary{positioning,calc,external}
%\tikzexternalize[prefix=figures/] 

\usepackage{crysymb}

\renewcommand*{\arraystretch}{1.2}

\usepackage{soul}
\usepackage[en-GB]{datetime2}
\DTMlangsetup[en-GB]{ord=omit}

\usetikzlibrary{positioning,calc}
\graphicspath{ {../../images/} }

\title{ICS0026 Cryptography}
\subtitle{Commitments \& identification}
\date{\DTMdate{2024-04-09}}
\author{Taaniel Kraavi}
\institute%
{%
  \textit{IT College}\\
  \textit{Tallinn University of Technology}
}

\begin{document}
\begin{frame}
  \titlepage
\end{frame}

\begin{frame}{Remote coin flip ver. 1}
  How can Alice \& Bob remotely decide an outcome via a coin flip?
  \begin{enumerate}[<+(1)->]
    \item Alice \enquote{calls} the result
    \item Bob flips the coin
    \item Alice wins if she guessed correctly, else Bob wins
  \end{enumerate}

  \vspace*{1em}

  \pause
  How can Alice fix a value without immediately revealing it? 
  \begin{itemize}[<+(1)->]
    \item If Alice cheats, Bob must be able to catch it
    \item Bob must not learn the value before Alice reveals it
  \end{itemize}
\end{frame}

\begin{frame}{Remote coin flip ver. 2}
  How can Alice and Bob jointly flip a coin remotely?
  \begin{itemize}[<+(1)->]
    \item Alice flips her coin and sends her result to Bob
    \item Bob flips his coin and sends his result to Alice
    \item Alice \& Bob XOR their results to obtain the same value
  \end{itemize}

  \vspace*{1em}

  \pause
  What additional challenges do you foresee?
\end{frame}

\begin{frame}{Commitment schemes}
  A \emph{commitment} fixes a value but allows it to be revealed later.
  \begin{itemize}[<+(1)->]
    \item A committer and a receiver
    \item The committer \emph{commits} their value
    \item The committer \emph{opens} (reveals/decommits) the value later (not always)
  \end{itemize}

  \vspace*{1em}

  \pause
  Two essential properties of commitment schemes:
  \begin{itemize}[<+(1)->]
    \item \emph{Hiding}: the commitment alone does not reveal the committed value
    \item \emph{Binding}: the committer cannot open the commitment to a different value
  \end{itemize}
\end{frame}

\begin{frame}{Quantifying the properties}
  Perfect security
  \begin{itemize}
    \item There is a zero chance that an unbounded adversary can break the property
  \end{itemize}

  \pause
  Statistical security
  \begin{itemize}
    \item There is a negligible chance that an unbounded adversary can break the property
  \end{itemize}

  \pause
  Computational security
  \begin{itemize}
    \item There is a negligible chance that a computationally bounded adversary can break the property
  \end{itemize}

  \pause
  Commitments cannot be both perfectly hiding and perfectly binding.
\end{frame}

\begin{frame}{Cryptosystems as commitment schemes}
  Any functional IND-CPA cryptosystem is hiding and perfectly binding.
  \begin{itemize}[<+(1)->]
    \item Intuitively, why is it perfectly binding?
    \item Why is it hiding?
    \item Why is it not perfectly hiding?
  \end{itemize}

  \vspace*{1em}

  \pause
  E.g. ElGamal encryption can be used as a commitment scheme.
  \begin{itemize}[<+(1)->]
    \item ElGamal commitments may refer to both classic and lifted ElGamal
    \item We can create public ElGamal parameters without knowing the secret key
    \item If the secret key is known, the scheme becomes \emph{extractable}
  \end{itemize}
\end{frame}

\begin{frame}{Two additional properties}
  Extractability
  \begin{itemize}[<+(1)->]
    \item Knowing a secret value, the message can be extracted from the commitment
    \item Public parameters are indistinguishable in both settings
  \end{itemize}

  \pause
  Equivocability
  \begin{itemize}[<+(1)->]
    \item \enquote{Fake} commitments are indistinguishable from real commitments
    \item the fake commitments can be opened to arbitrary values
  \end{itemize}

  \vspace*{1em}

  \pause
  Some commitments therefore require a trusted setup.
\end{frame}

\begin{frame}{Pedersen commitment}
  Let $q$ be prime and let $\GG = \langle g \rangle$ be a $q$-element DL-group.\\
  Choose $h$ uniformly from $\GG\setminus\{1\}$ and set $\PK \gets (g, h)$.

  \vfill

  \pause
  To commit $m \in \ZZ_q$, choose $r \getsu \ZZ_q$ and output
  \[
    \begin{cases}
      c \gets g^m h^r\ ,\\
      d \gets (m, r)\ .
    \end{cases}
  \]

  \vfill

  \pause
  A tuple $(c, m, r)$ is a valid decommitment for $m$ if $c = g^m h^r$.
\end{frame}

\begin{frame}{Pedersen commitment}
  Pedersen commitments are
  \begin{itemize}[<+(1)->]
    \item Perfectly hiding
    \item Computationally binding
    \item Equivocable
    \begin{itemize}
      \item Hint: What happens if the relation between $g$ and $h$ is known?
    \end{itemize}
  \end{itemize}

  \vspace*{1em}

  \pause
  Both Pedersen and ElGamal commitments are homomorphic.
  \begin{itemize}
    \item Only lifted ElGamal commitments are \enquote{interoperable} with Pedersen's.
  \end{itemize}
\end{frame}

\begin{frame}{Why do we need commitments?}
  Commitments are an essential primitive in cryptography.
  \begin{itemize}[<+(1)->]
    \item Typically they are not used on their own (although they can)
    \item It is possible to prove relations over committed values
    \item Commitments are essential to many zero-knowledge protocols
    \item More generally, many secure protocols rely on commitments
  \end{itemize}

  \vspace*{1em}

  \pause
  How would you use commitments in the remote coin flipping protocol?
\end{frame}

\begin{frame}{Identification protocols}
  \pause
  Identification protocols are generally based on one or more of:
  \begin{itemize}[<+(1)->]
    \item What you are (biometrics)
    \item What you have
    \item What you know
  \end{itemize}

  \vspace*{1em}

  \pause
  Cryptography works with the latter two.
  \begin{itemize}[<+(1)->]
    \item What you know: some secret
    \item What you have: cryptographic hardware device (last week's topics)
  \end{itemize}

  \vspace*{1em}

  \pause
  Identification requires \emph{freshness}, but does not require \emph{non-repudiation}.
\end{frame}

\begin{frame}{Security assumptions}
  \pause
  We must assume a trusted registration protocol
  \begin{itemize}[<+(1)->]
    \item Digital identification is not possible without it
    \item No legal identity without physical measures
  \end{itemize}

  \vspace*{1em}

  \pause
  We must assume an authenticated communication channel
  \begin{itemize}[<+(1)->]
    \item An adversary may otherwise MITM the protocol
  \end{itemize}

  \vspace*{1em}

  \pause
  We only consider cryptographic attacks
  \begin{itemize}[<+(1)->]
    \item Cryptography cannot defend against \enquote{non-math} attacks
    \item Physical attacks against smart cards, credential exfiltration, \dots
  \end{itemize}
\end{frame}

\begin{frame}{Challenge-response paradigm}
  A prover must answer a verifier's challenge for identification.

  \vspace*{1em}

  \pause
  Familiar approaches:
  \begin{columns}[T,onlytextwidth]
    \begin{column}{0.5\textwidth}
      \begin{itemize}[<+(1)->]
        \item Symmetric encryption
        \begin{itemize}
          \item KPA, CPA2
        \end{itemize}
        \item Symmetric authentication (HMAC)
        \begin{itemize}
          \item ROM
        \end{itemize}
      \end{itemize}
    \end{column}
    \begin{column}{0.5\textwidth}
      \begin{itemize}[<+(1)->]
        \item Asymmetric encryption
        \begin{itemize}
          \item CPA, CCA2
        \end{itemize}
        \item Asymmetric authentication
        \begin{itemize}
          \item KMA, CCA2
        \end{itemize}
      \end{itemize}
    \end{column}
  \end{columns}

  \vspace*{1.5em}

  \pause
  What about password-based authentication?
  \begin{itemize}[<+(1)->]
    \item The symmetric versions are only practical with proofs of knowledge
    \item For proof of possession, asymmetric versions must be used
  \end{itemize}
\end{frame}

\begin{frame}{Zero-knowledge identification}
  IND-CCA2 is quite \enquote{expensive} in practice.
  \begin{itemize}[<+(1)->]
    \item Challenge-response protocols based on asymmetric schemes are therefore not necessarily efficient.
  \end{itemize}

  \vspace*{1em}

  \pause
  How can we prevent cheating verifier attacks with an \enquote{efficient} scheme?
  \begin{itemize}[<+(1)->]
    \item A cheating verifier must not get useful information from the prover
    \item The protocol must be \emph{zero-knowledge}
    \item The protocol must convince an honest verifier that an (honest) prover knows the secret
  \end{itemize}
\end{frame}

\begin{frame}{Schnorr protocol}
  Schnorr's protocol allows proving knowledge of a discrete logarithm.

  \pause
  \begin{center}
    \begin{tikzpicture}
      \node (alice) at (0,0) {\includegraphics[height=80px]{alice}};
      \node (bob)   at (6,0) {\includegraphics[height=80px]{bob}};

      \node(sk) at (alice.north) [yshift=10px] {$x \in \ZZ_q$};
      \node(pk) at (bob.north)   [yshift=10px] {$h = g^x$};

      \node (n1a) at (alice.east) [yshift=18px] {};
      \node (n1b) at (bob.west)   [yshift=18px] {};
      \node (n2a) at (alice.east) {};
      \node (n2b) at (bob.west)   {};
      \node (n3a) at (alice.east) [yshift=-18px] {};
      \node (n3b) at (bob.west)   [yshift=-18px] {};

      \node at (alice.west) [left,yshift=20px]  {$r \getsu \ZZ_q$};
      \node at (bob.east)   [right,yshift=20px] {$c \getsu \ZZ_q$};

      \draw[->,thick] (n1a) -- (n1b) node[above,midway] {$u = g^r$};
      \draw[<-,thick] (n2a) -- (n2b) node[above,midway] {$c$};
      \draw[->,thick] (n3a) -- (n3b) node[above,midway] {$z = r + cx$};

      \node at (bob.south) [yshift=-10px] {$g^z = g^r g^{cx} \iseq u h^c$};
    \end{tikzpicture}
  \end{center}

  The group $\GG = \langle g\rangle$ must be a DL group with a prime cardinality $q$.
\end{frame}

\begin{frame}{Sigma protocols}
  \pause
  $\Sigma$-protocols are efficiently computable three-move protocols that must be (special) honest verifier zero-knowledge.

  \pause
  \begin{center}
    \begin{tikzpicture}
      \node (alice) at (0,0) {\includegraphics[height=80px]{alice}};
      \node (bob)   at (6,0) {\includegraphics[height=80px]{bob}};

      \node(rand) at (alice.north) [yshift=10px] {$r \getsu \RRR$};
      \node(chal) at (bob.north)   [yshift=10px] {$c \getsu \CCC$};

      \node (n1a) at (alice.east) [yshift=18px] {};
      \node (n1b) at (bob.west)   [yshift=18px] {};
      \node (n2a) at (alice.east) {};
      \node (n2b) at (bob.west)   {};
      \node (n3a) at (alice.east) [yshift=-18px] {};
      \node (n3b) at (bob.west)   [yshift=-18px] {};

      \draw[->,thick] (n1a) -- (n1b) node[above,midway] {$u$};
      \draw[<-,thick] (n2a) -- (n2b) node[above,midway] {$c$};
      \draw[->,thick] (n3a) -- (n3b) node[above,midway] {$z = z_\SK(u,c)$};

      \node at (alice.south) [yshift=-10px] {$\SK$};
      \node at (bob.south)   [yshift=-10px] {$\VVV_\PK(u, c, z)$};
    \end{tikzpicture}
  \end{center}

  \pause
  In the \emph{transcript} $(u, c, v)$, $u$ is a commitment, $c$ is a (varying) challenge, and $z$ is a (consistent) response.
\end{frame}

\begin{frame}{Danger zone}
  Sigma protocols \& zero-knowledge schemes are advanced cryptography.
  \begin{itemize}[<+(1)->]
    \item You should know that they exist
    \item You should not use them directly
    \item The complexity snowballs
    \item Not comparable to popular and standardised schemes
    \begin{itemize}
      \item encryption \& signature schemes, KEMs, MACs, hash functions
    \end{itemize}
  \end{itemize}

  \vspace*{1em}

  \pause
  These topics are a prime example of unknown unknowns.
  \begin{itemize}[<+(1)->]
    \item Reading a few blog posts to get an intuition to start using them is not enough
    \item Extensive amounts of study required for even the \enquote{entry level}
    \item Not gatekeeping:
    you are more than welcome to spiral down the rabbit hole
  \end{itemize}
\end{frame}

\end{document}
