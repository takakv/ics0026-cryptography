\documentclass{practice}

\title{7}
\date{\today}

%\usepackage{etoolbox}
%\makeatletter
%\preto{\@verbatim}{\topsep=0pt \partopsep=-2\parsep }
%\makeatother

\usepackage{crysymb}

\usepackage{booktabs}

\usepackage{fancyvrb}
\fvset{listparameters=\setlength{\topsep}{0pt}\setlength{\partopsep}{0pt}}

\usepackage{listings}

\lstset{
basicstyle=\small\ttfamily,
columns=flexible,
breaklines=true
}

\begin{document}
\maketitle

\begin{task}{PKCS v1.5 vs. PSS}
  \textit{Task.}
  For this task, use PyCryptodome to generate two RSA signatures in PKCS v1.5 mode on the same message, and compare the two signatures.
  Next, generate two signatures on the same message in PSS mode and compare the signatures.
  
  Do your observations correspond with what we learnt in class?
\end{task}

\begin{task}{RSA implementations}
  \textit{Task.}
  Check out \href{https://nvd.nist.gov/vuln/detail/CVE-2016-1494}{CVE-2016-1494} which is about using the \href{https://www.darkreading.com/cyberattacks-data-breaches/-berserk-bug-uncovered-in-mozilla-nss-crypto-library-impacts-firefox-chrome}{\textit{BERserk}} attack against the \href{https://github.com/sybrenstuvel/python-rsa}{\texttt{rsa}} Python library to forge signatures.
  If you are interested, you can read more about it in the corresponding \href{https://words.filippo.io/bleichenbacher-06-signature-forgery-in-python-rsa/}{\textit{blog post}}.

  In class, I mentioned that RSA PKCS v1.5 signatures have no known weaknesses.
  So how come this attack exists?\footnote{The attack is feasible because of an implementation error, not a crypto-theoretical error.
  Nevertheless, it highlights why you should prefer PSS padding.} 

  Search to see whether PyCryptodome has had any RSA-related vulnerabilities reported in the past.

  \begin{tcolorbox}[title=Warning]
    Even though the BERserk vulnerability is Python-RSA is now \href{https://github.com/sybrenstuvel/python-rsa/commit/ab5d21c3b554f926d51ff3ad9c794bcf32e95b3c}{\textit{fixed}}, there are still other issues with it such as side-channel vulnerabilities\footnote{\url{https://github.com/sybrenstuvel/python-rsa/issues/169}}\footnote{\url{https://github.com/sybrenstuvel/python-rsa/issues/165}} caused by it being a pure-python implementation.
    It also does not support PKCS \#1 v2.0 and later.
    \tcblower
    Do not use Python-RSA if alternatives are available.
    In general, do not use pure-python crypto implementations.
  \end{tcolorbox}
\end{task}

\begin{task}{PyNaCl}
  \textit{Preface.}
  Enough of the PyCryptodome package, time to use other tools too!
  \href{https://pynacl.readthedocs.io/en/latest/}{\emph{PyNaCl}} is a Python module that provides bindings to \href{https://doc.libsodium.org}{\emph{libsodium}} (called \emph{Sodium}), a neat and fairly lightweight cryptographic library.
  
  Sodium is based on the \href{http://nacl.cr.yp.to}{\emph{NaCl: Networking and Cryptography library}} (called `salt' in spoken language) developed by Bernstein et al.
  While Sodium is still compatible with the NaCl API, it has also implemented new algorithms that are not available in NaCl.
  These days, you should prefer Sodium unless you have a good reason to use NaCl.

  Sodium is also an opinionated library, meaning that it makes some decisions\footnote{\url{https://doc.libsodium.org/internals\#security-first}} to avoid letting users shoot themselves in the foot, even if it impacts speed.
  However, it cannot of course protect against every possible mistake one could make, so you must still take care when using it.

  \textit{Task.}
  For this task, generate an Ed25519 signing key with \href{https://pynacl.readthedocs.io/en/latest/signing/#}{\texttt{pynacl}}.
  Next, sign some messages and verify their signatures.
  Try also verifying signatures for wrong messages.

  Finally, benchmark the time it takes to generate $1000$ signatures for the same message.
\end{task}

\begin{task}{Meet your peers}
  \textit{Preface.}
  Generating and verifying signatures for keys you have generated on your own becomes boring really fast.
  In the real world, usually someone else verifies your signature,
  or you have to verify someone else's signature.

  While it was important to introduce PyNaCl with the previous task, it does not offer convenient key management like PyCryptodome does.

  Note that EdDSA offers two `modes' (not comparable with block cipher modes):
  \begin{itemize}
    \item PureEdDSA (Ed25519, Ed448): the message is fed to the signing algorithm directly.
    \item HashEdDSA (Ed25519ph, Ed448ph): the message is pre-hashed and the hash is fed to the signing algorithm.
    
    This has some advantages (e.g. for memory constrained devices) but also disadvantages, such as reduced collision resistance.
  \end{itemize}
  You should always use PureEdDSA over HashEdDSA unless you have a very good reason to use the latter one.

  \textit{Task.}
  Using \href{https://pycryptodome.readthedocs.io/en/latest/src/signature/eddsa.html}{\texttt{pycryptodome}}, generate an Ed448 keypair.
  Exchange your public key and a signed message (you can try to cheat) with your neighbour or another classmate of your choice.
  Finally, verify their signature on their message.
  Did they give you the correct message/signature?
\end{task}

\begin{task}{SPHINCS+}
  \textit{Preface.}
  NIST has picked SPHINCS\textsuperscript{+} as one of the three post-quantum signature algorithms that it will standardise as part of the move towards PQC.
  Out of these three algorithms, SPHINCS\textsuperscript{+} is the only hash-based one.
  The other two are lattice-based.

  In turn, SPHINCS+ is divided into three schemes:
  \begin{itemize}
    \item SPHINCS\textsuperscript{+}-SHAKE256
    \item SPHINCS\textsuperscript{+}-SHA-256
    \item SPHINCS\textsuperscript{+}-Haraka
  \end{itemize}
  which differ in the hash function the scheme is instantiated with.

  Furthermore, the schemes are split into the \texttt{f} and \texttt{s} versions, which can roughly be summarised as:
  \begin{itemize}
    \item \texttt{f}: fast signing but large signatures,
    \item \texttt{s}: smaller signatures, but slower signing.
  \end{itemize}

  Finally, SPHINCS\textsuperscript{+} offers three different security levels:
  \begin{center}
    \begin{tabular}{@{}l r r r r@{}}
      \toprule
      Algorithm & bits of security & $\PK$ bytes & $\SK$ bytes & $\sigma$ bytes\\
      \midrule
      \texttt{SPHINCS\textsuperscript{+}-128s} & 133 & 32 &  64 &  7 856\\
      \texttt{SPHINCS\textsuperscript{+}-128f} & 128 & 32 &  64 & 17 088\\
      \texttt{SPHINCS\textsuperscript{+}-192s} & 193 & 48 &  96 & 16 224\\
      \texttt{SPHINCS\textsuperscript{+}-192f} & 194 & 48 &  96 & 35 664\\
      \texttt{SPHINCS\textsuperscript{+}-256s} & 255 & 64 & 128 & 29 792\\
      \texttt{SPHINCS\textsuperscript{+}-256f} & 255 & 64 & 128 & 49 856\\
      \bottomrule
    \end{tabular}
  \end{center}
  These values are taken from the reference implementation's GitHub \href{https://github.com/sphincs/sphincsplus}{\textit{repo}}.
  A nice article (nice tables) on the parameter sets and performance is ePrint \href{https://eprint.iacr.org/2022/1725}{\texttt{2022/1725}}.

  The `official' SPHINCS\textsuperscript{+} signature library is provided by the Python module \href{https://github.com/sphincs/pyspx}{\texttt{pyspx}}.
  That is, it provides bindings to the reference C implementation of SPHINCS\textsuperscript{+}.

  \textit{Task.}
  For this task, generate a \texttt{SPHINCS\textsuperscript{+}-128f} signature with the SHAKE256 hash algorithm.
  How long did signing take?

  Then, generate a signature with the \texttt{s} version.
  How long did signing take now?

  Next, benchmark the time needed to generate $1000$ signatures with the \texttt{f} version.
  Compare this time with the time taken for $1000$ Ed25519 signatures.
  How could you make the comparison `fairer'?
\end{task}

\begin{task}{EdDSA with OpenSSL}
  \textit{Preface.}
  At this point, you should have some familiarity with OpenSSL.
  The \href{https://www.openssl.org/docs/man3.0/man1/openssl-dgst.html}{\texttt{dgst}} tool is typically used for signing (with pre-hashing) and for verifying signatures.
  However, the \texttt{dgst} tool does not support EdDSA signing.
  Instead, you should use the \href{https://www.openssl.org/docs/man3.0/man1/openssl-pkeyutl.html}{\texttt{pkeyutl}} tool.

  \textit{Task.}
  For this task, familiarise yourself with the OpenSSL documentation.
  Then, generate an Ed25519 keypair with OpenSSL and try to sign and verify a signature from the command line.

  Verify the signature with some other tool (e.g. in Python).
  Did OpenSSL use PureEdDSA?

  This task is cursed and will prove to you that OpenSSL (and especially its documentation) is horrible.
  This task will likely be part of the homework as well ;).
\end{task}

\end{document}
