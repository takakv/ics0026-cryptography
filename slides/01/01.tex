\usepackage{graphicx}
\usepackage{soul}

\title{ICS0026 Cryptography}
\subtitle{Introduction to cryptography}
\date{\today}
\author{Taaniel Kraavi}
\institute%
{%
  \textit{IT College}\\
  \textit{Tallinn University of Technology}
}

\begin{document}
\begin{frame}[plain]
  \titlepage
\end{frame}

\begin{frame}
  \frametitle{What's in a name?}

  \pause
  \begin{itemize}[<+->]
    \item Historically: the art of codes
    \item Cornerstone of computer and information security
    \item Replace a hard problem with mathematics
    \item Cryptography $\neq$ cryptocurrency / blockchain
  \end{itemize}
\end{frame}

\begin{frame}
  \frametitle{CIA and cryptography}

  \pause
  \begin{itemize}[<+(1)->]
    \item Confidentiality
    \item Integrity
    \item \st{Availability}
    \pause
    \item Authenticity
    \item Non-repudiation
  \end{itemize}
\end{frame}

\begin{frame}
  \frametitle{Ladder of abstraction}

  \pause
  \begin{itemize}
    \item \textbf{Many levels of building blocks}
    \pause
    \begin{itemize}
      \item Mathematical foundations
      \item Cryptographic primitives
      \item Cryptographic protocols
    \end{itemize}

    \vspace*{1em}

    \pause
    \item \textbf{Implementation details obscure security}
    \pause
    \begin{itemize}
      \item Abstractions highlight desirable properties.
      \item Some properties are \enquote{relevant} for security.
      \item Security is formally defined through these properties.
    \end{itemize}
  \end{itemize}

  \vspace*{1em}

  \pause
  Implementation security is an additional layer.
\end{frame}

\begin{frame}
  \frametitle{Defining security}

  \pause
  \begin{itemize}
    \item \textbf{Two approaches to security}
    \begin{itemize}[<+(1)->]
      \item Heuristic security / best-effort design
      \item Provable security
    \end{itemize}

    \vspace*{1em}

    \pause
    \item \textbf{Two types of security}
    \begin{itemize}[<+(1)->]
      \item Conditional
      \item Unconditional\textsuperscript{*}
    \end{itemize}
  \end{itemize}

  \vspace*{1em}

  \pause
  \textbf{Caveat:} Even unconditional security is conditional.
  \pause
  \begin{itemize}
    \item A \emph{trust model} must be specified.
    \pause
    \item Proper procedure must be followed.
  \end{itemize}
\end{frame}

\begin{frame}
  \frametitle{Kerckhoff's principle}

  \pause
  \textbf{Definition.}
  A cryptographic system should be secure even if everything about the system, except the key, is public knowledge.

  \vspace*{1em}

  \pause
  \textbf{Interpretation.}
  A system's security should depend on the secrecy of its usage parameters, not of its algorithms.
\end{frame}

\begin{frame}
  \frametitle{Intractability}

  \begin{itemize}
    \item \textbf{Cryptographic constructions are rarely provably secure.}
    \pause
    \begin{itemize}[<+(1)->]
      \item There is an unavoidable gap between theory and practice.
      \item Practical security is based on \emph{intractability}.
      \item It is \emph{infeasible} in practice to solve a problem.
    \end{itemize}

    \vspace*{1em}

    \pause
    \item In the physical world, we must obey the laws of physics.
    \begin{itemize}[<+(1)->]
      \item Humans keep pushing what is feasible.
      \item Dedicated tooling improves attack success.
    \end{itemize}
    
    \vspace*{1em}

    \pause
    \item Never assume secure parameters: verify with experts.
    \begin{itemize}[<+(1)->]
      \item Recommendations by ANSSI, BSI, NIST\textsuperscript{*}.
      \item Independent opinions of respected cryptographers.
      \item Err on the side of caution.
      \item What is safe today may not be safe soon.
    \end{itemize}
  \end{itemize}
\end{frame}

\begin{frame}
  \frametitle{Cryptosystems}
  \framesubtitle{Informal introduction}

  \pause
  Let $m$ be some message called the \emph{plaintext}.

  \vspace*{1em}

  \pause
  A cryptosystem is a triple of algorithms $(\GEN,\ENC,\DEC)$ where
  \begin{itemize}[<+(1)->]
    \item $\GEN$ is a key generation algorithm
    \item $\ENC$ is an encryption algorithm: $c\gets\ENC(m)$
    \item $\DEC$ is a decryption algorithm: $m\gets\DEC(c)$
  \end{itemize}

  \vspace*{1em}

  \pause
  The output of the encryption $c\gets\ENC(m)$ is called the \emph{ciphertext}.

  \vspace*{1em}

  \pause
  The cryptosystem is functional if $\DEC(\ENC(m))=m$.
\end{frame}

\begin{frame}[c]
  \centering
  What makes a cryptosystem secure?
\end{frame}

\begin{frame}
  \frametitle{Attack types}

  \begin{itemize}[<+(1)->]
    \item Ciphertext-only attack
    \begin{itemize}
      \item You only know a ciphertext.
    \end{itemize}
    \item Known-plaintext attack (KPA)
    \begin{itemize}
      \item You are given a ciphertext--plaintext pair.
    \end{itemize}
    \item Chosen-plaintext attack (CPA)
    \begin{itemize}
      \item You are given ciphertexts for plaintexts you provide.
    \end{itemize}
    \item Chosen-ciphertext attack (CCA)
    \begin{itemize}
      \item You are given plaintexts for ciphertexts you provide.
    \end{itemize}
  \end{itemize}
\end{frame}

\begin{frame}
  \frametitle{Simple ciphers}

  \pause
  \begin{itemize}
    \item Substitution ciphers (monoalphabetic)
    \begin{itemize}
      \item shift cipher
      \item substitution cipher
    \end{itemize}
    \pause
    \item Substitution ciphers (polyalphabetic)
    \begin{itemize}
      \item Vigenère cipher
    \end{itemize}
    \pause
    \item Transposition ciphers
    \begin{itemize}
      \item permutation cipher
    \end{itemize}
  \end{itemize}
\end{frame}

\begin{frame}
  \frametitle{Security properties}

  \pause
  \begin{itemize}
    \item Defined by \emph{Claude Shannon}.
    
    \vspace*{1em}

    \pause
    \item \textbf{Confusion}
    \begin{itemize}
      \item Obfuscates the relation between ciphertext \& key.
      \item Difficulty to extract key even with CCA.
    \end{itemize}

    \vspace*{1em}

    \pause
    \item \textbf{Diffusion}
    \begin{itemize}
      \item Obfuscates the relation between plaintext \& ciphertext.
      \item Difficulty to conduct meaningful ciphertext-only attacks.
    \end{itemize}

    \vspace*{1em}

    \pause
    \item A good cryptosystem should use both.
  \end{itemize}
\end{frame}

\end{document}
