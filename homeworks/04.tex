\documentclass{homework}

\title{04}
\date{\DTMdate{2024-12-06}}

\usepackage{crysymb}

\begin{document}
\maketitle

\textbf{Reminders}

\begin{itemize}
  \item Include in your report approximately how long the homework took for you.
  \item Read through task descriptions carefully.
  \item You may request deadline extensions until 03.12.24 23:59 EET.
  \item Deadlines are strict: if you do not submit your assignment by the deadline, you fail it.
  \item To pass the course you must pass all homework assignments.
  \item Second chances:
  \begin{itemize}
    \item If I review your assignment before the deadline, I will \emph{always} give you a chance to fix it for the deadline (or with a slight extension).
    Start early!
    \item If I review your assignment after the deadline, I will give you an extension \emph{only} if you have explained what part of your code does not work, and what you did to attempt to fix it.
    Ask questions when you are stuck to avoid risking it!
    \item If I give you an extension, you must still fix your homework by the extended deadline to pass.
    \item If your code does not work but you are not aware of it or have not explained your issues, you fail the homework (if I review it after the deadline).
    Test your code!
    \item If you fail your homework, you can appeal my decision within 3 days of receiving your grade (grade day + 3).
    Write me an email or message me on Discord.
    I will then organise an oral defence for you.
  \end{itemize}
  \item You may share hints with each-other, but no answers/code.
  \item You may ask me questions (I prefer Discord to Moodle/email).
  If your code does not work and you have not asked me any questions, you will have a hard time arguing that you actually tried solving it!
  \item If I catch you plagiarising/copying, I will report you to the faculty and fail you for the course.
\end{itemize}

\newpage

\begin{center}
  Theory tasks
\end{center}

\textbf{Additional instructions}

\begin{itemize}
  \item Submit as a PDF file called \texttt{studentid-04.pdf} where \enquote{studentid} is your 6 letter student ID/username (\enquote{takraa} for me).
  \item \textbf{Do not copy paste answers.}
  I am not expecting perfect answers, rather I want to see your reasoning, and how you have understood these concepts.
  As long as your answers show reasoning, I will accept them, even if they are not correct.
  This applies to all of the theory tasks.
\end{itemize}

\begin{task}{Proof systems \& zero-knowledge}
  For this task, answer the following questions using complete sentences:
  \begin{itemize}
    \item Why should a proof system be sound?

    \item Are interactive proofs typically transferable?
    That is, should an observer who observes an interactive proof but trusts neither the prover nor the verifier be convinced by the proof?

    \item What mechanism can be used to transform an interactive proof into a non-interactive one?
    Do the security guarantees remain the same?

    \item Consider an interactive proof system that provides $256$ bits of security.
    Could you maintain this security level by using SHA-256 in the non-interactive version of the proof?
    Why yes/why not?
  \end{itemize}

  Then, describe a use case where a zero-knowledge proof would be useful in practice, which is not one of the examples discussed in class (age range, citizenship of member states, online voting).

  If you used online materials to help you answer these questions, please include references in your answers.
\end{task}

\begin{task}{TLS}
  For this task, answer the following questions:
  \begin{itemize}
    \item Describe three aspects in which TLS 1.3 improves on TLS 1.2.
    Explain in your own words why you think these are important improvements.

    \item Pick a website that uses HTTPS, and answer the following (include screenshots):
    \begin{itemize}
      \item What is the website link?
      \item Is the certificate domain validated, organisation validated, or extended validated?
      How do you know?
      \item What is the validity period of the certificate?
      \item Where is the OCSP responder or CRL?
      \item What is the physical address of the root CA?
    \end{itemize}

    \item Why do we need certificate transparency?
  \end{itemize}

  If you used online materials to help you answer these questions, please include references in your answers.
\end{task}

\begin{task}{A taste of the exam}
  For this task, answer at least 10 of following questions using complete sentences.
  I encourage you to answer more, but 10 is the minimum.

  Be succinct and precise: no word soup!
  In the exam, if you are not able to give a clear answer, I will consider that you do not know the answer.

  \begin{enumerate}
    \item Does cryptography protect the availability of information?
    Why yes/why not?
    \item Why do we need encryption?
    \item Why is IND-CPA important?
    \item Why do MACs not offer non-repudiation?
    \item Why is public key crypto not safe against unbounded adversaries?
    \item Does second pre-image resistance imply collision resistance?
    \item Why is using a cryptographic hash function for hashing passwords a bad idea?
    \item What is important when protecting passwords?
    \item What is one role of a key derivation function?
    \item What are block cipher modes?
    \item What is the advantage of AES-GCM over AES-CTR (or of ChaCha-Poly over ChaCha)?
    \item What do digital signatures provide?
    \item What is typically needed on top of digital signatures for trust?
    \item Why do we need eIDAS?
    \item How is the authenticity of a server checked when you visit a website?
    \item How do devices/browsers trust root CAs?
    \item What is the role of a CA?
    \item What is needed for the OTP to be unconditionally secure?
    \item Why is the OTP unconditionally secure?
    \item What are some advantages of OCSP over CRLs?
    \item What are the two core properties of a commitment scheme?
    \item Why does an identification protocol require \enquote{freshness}?
  \end{enumerate}

  These questions you should be able to answer on your own.
  If you need to look up the answers, then that is a good indicator that you will need to prepare for the exam and go through materials and/or recordings again.
\end{task}

\begin{task}{Your thoughts (optional)}
  This is an \emph{optional} task, but I'd appreciate it if you at least answer the first question.
  You can also choose to answer only some of the questions.
  It's not anonymous (clearly) but this is because I might want to ask you questions if you say something particularly intriguing.
  It's fine to say negative things about the course and/or me, I only ask for the answer to be constructive.

  \begin{itemize}
    \item What's your mood? (select up to 2)
    \begin{enumerate}
      \item You killed crypto for me, I hope to never deal with it!
      \item I liked crypto more before this class.
      \item It was ok, but that's the last I'll deal with it.
      \item It was interesting, but that's the last I'll deal with it.
      \item I like crypto more after this class.
      \item It was interesting, I may pursue crypto in the future.
      \item I aspire to be the next Daniel Bernstein.
    \end{enumerate}
    \item Do you feel like this course has been useful for you? (it's okay to say no)
    \item Do you feel like this course has been fair? (again it's ok to say no)
    \item What has/have been the hardest topic(s)?
    \item What would you have liked to be different?
    \item What would you have liked to see more/less of?
    \item Any additional thoughts/suggestions for future iterations of this class?
  \end{itemize}
\end{task}

\newpage
\setcounter{task}{0}

\begin{center}
  Practical tasks
\end{center}

\begin{task}{Git}
  For this task, you will need either a GitHub or a TalTech GitLab account\footnote{\url{http://gitlab.cs.taltech.ee}}.
  Once you have set up an account, create a repository called \texttt{ics0026}, and make sure that it is publicly visible (GitHub) or internally visible (TalTech GitLab).
  If you want to make it privately visible, you must add me to your repository (\texttt{@takakv} on GitHub, \texttt{@takraa} on TalTech GitLab).

  Then, you must set up a commit signing key.
  I recommend that you use an SSH key for signing commits, but you can also use GPG or S/MIME.
  I will not include a reference here: you should be able to find one yourself (GitHub and GitLab both have their own articles for this).

  For the task, you must push a signed commit into your \texttt{ics0026} repository.
  This commit should include the public key of the key you use for signing: call it simply \texttt{key}.
  The key should not be a binary blob: it should open in a text editor (e.g. PEM).
  Your task is considered complete if the GUI shows that your commit is verified.

  \begin{figure}[h!]
    \center
    \includegraphics[width=\textwidth]{../images/gitlab_verified.png}
    \caption{Verified commit on GitLab.}
  \end{figure}

  \begin{figure}[h!]
    \center
    \includegraphics[width=\textwidth]{../images/github_verified.png}
    \caption{Verified commit on GitHub.}
  \end{figure}

  Include the repository URL in your report.
\end{task}

\begin{task}{SSH \& SCP}
  For this task, you must SSH into \texttt{enos.itcollege.ee} using an SSH key.

  If you have not yet set up an SSH key, you will need to get it onto the server, e.g. with \texttt{ssh-copy-id}\footnote{\url{https://www.ssh.com/academy/ssh/copy-id\#copy-the-key-to-a-server}}.
  The connection will ask you for your account password on your connection: this is your UniID password.

  Once you can SSH into Enos using an SSH key, take a screenshot of the log in flow.
  If your SSH connection does not prompt you for a password, explain in writing how you have protected your key (e.g. password in SSH agent, SSH key on Yubikey).
  For example, my SSH key is a non-resident key on my Yubikey, and the SSH key password is automatically loaded by my SSH agent.

  \begin{figure}[h!]
    \center
    \includegraphics[width=\textwidth]{../images/enos_ssh.png}
  \end{figure}

  Once in the server, create an empty directory called \texttt{ics0026} in your \texttt{public\_html} folder.
  Then, close the connection with \texttt{exit}.

  Upload the screenshot that you took fo the log in flow into the \texttt{\textasciitilde/public\_html/ics0026} folder on the server using SCP (secure copy)\footnotemark{}.
  Take a screenshot of this process as well, and add it also to the folder on the server.
  \footnotetext{\url{https://linuxize.com/post/how-to-use-scp-command-to-securely-transfer-files/}}

  Finally, include \emph{both} screenshots also in your report.
  To verify whether your screenshots are correctly on the server, visit \url{https://enos.itcollege.ee/~takraa/} but with your own student username instead of \texttt{takraa}.
\end{task}
\end{document}
