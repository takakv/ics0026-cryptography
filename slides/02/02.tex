\usetheme[numbering=fraction]{metropolis}

\usepackage{cryslides}
\usepackage{crygame}

\usepackage{soul}

\usetikzlibrary{positioning,calc}

\graphicspath{ {../../images/} }

\title{ICS0026 Cryptography}
\subtitle{Randomness, OTP, and stream ciphers}
\date{\DTMdate{2024-09-10}}
\author{Taaniel Kraavi}
\institute%
{%
  \textit{IT College}\\
  \textit{Tallinn University of Technology}
}

\begin{document}
\begin{frame}[plain]
  \titlepage
\end{frame}

\begin{frame}
  \frametitle{Is it random?}

  \begin{itemize}
    \item<1-> 2204219747
    \begin{itemize}
      \item<2-> \texttt{10000011 01100001 10111001 01100011}
      \item<2-> \texttt{0}: $17$, \texttt{1}: $15$
      \item<3-> Source: \texttt{/dev/urandom}
    \end{itemize}
    \item<1-> 1415926535
    \begin{itemize}
      \item<2-> \texttt{01010100 01100101 01010011 00000111}
      \item<2-> \texttt{0}: $18$, \texttt{1}: $14$
      \item<4-> Source: first 10 decimal digits of $\pi$
    \end{itemize}
    \item<1-> 1918987876
    \begin{itemize}
      \item<2-> \texttt{01110010 01100001 01101110 01100100}
      \item<2-> \texttt{0}: $17$, \texttt{1}: $15$
      \item<5-> Source: decoded ASCII string \enquote{rand}
    \end{itemize}
  \end{itemize}
\end{frame}

\begin{frame}
  \frametitle{Randomness}

  % Mention double pendulums, Cloudflare lava lamps
  \begin{itemize}[<+->]
    \item Randomness is unprovable
    \item Statistical tests for non-randomness
    \item \enquote{True} randomness:
    \begin{itemize}
      \item Thermal noise, photoelectric effect, quantum phenomena, \dots
    \end{itemize}
    \vspace*{1em}
    \item Random sequence
    \begin{itemize}
      \item forward \& backward unpredictable
      \item follows no deterministic pattern
      \item cannot be compressed
      \item has no description shorter than itself
    \end{itemize}
    \vspace*{1em}
    \item Computers and humans are predictable
  \end{itemize}
\end{frame}

\begin{frame}
  \frametitle{RNG}

  \begin{itemize}[<+->]
    \item Random number generators
    \item RNG boogaloo:
    \begin{itemize}
      \item hardware, true \& quantum RNG: HRNG, TRNG, QRNG
      \item NRBG: non-deterministic random bit generator
    \end{itemize}
    \item HRNG requires a physical source of entropy
    \begin{itemize}
      \item Integrated (opaque)
      \item External (more choice)
    \end{itemize}
    \item How do computers fill the entropy pool?
    \begin{itemize}
      \item Hardware sources, interrupt signals, user behaviour, \dots
      \item Beware of boot: low entropy
      \item Quantity of entropy?
      \item To block or not to block?
    \end{itemize}
    \item Do we always need a TRNG?
  \end{itemize}
\end{frame}

\begin{frame}
  \frametitle{PRNG}

  \begin{itemize}[<+->]
    \item Pseudo-random number generator
    \item PRNG boogaloo:
    \begin{itemize}
      \item CSPRNG: cryptographically secure PRNG
      \item DRBG: deterministic random bit generator
    \end{itemize}
    \item Deterministic
    \item Random input \emph{seed}
  \end{itemize}

  \vspace*{1em}

  \pause
  Some recommendations/standards:
  \begin{itemize}
    \item \href{https://csrc.nist.gov/projects/random-bit-generation}{NIST on random bit generation}
    \begin{itemize}
      \item \url{https://csrc.nist.gov/projects/random-bit-generation}
    \end{itemize}
    \item \href{https://www.bsi.bund.de/dok/randomnumbergenerators}{BSI on random number generators}
    \begin{itemize}
      \item \url{https://www.bsi.bund.de/dok/randomnumbergenerators}
    \end{itemize}
  \end{itemize}
\end{frame}

\begin{frame}
  \frametitle{An intuition on construction}

  We want a black box with:
  \begin{itemize}[<+(1)->]
    \item confusion
    \item diffusion (avalanche effect)
  \end{itemize}

  \pause
  Given this black box:
  \begin{itemize}[<+(1)->]
    \item compute one output \enquote{block} from the seed and counter
    \item increment the counter
    \item repeat as needed
  \end{itemize}

  \pause
  This is one type of construction, but there are others.
\end{frame}

\begin{frame}
  \frametitle{Getting randomness on *nix}

  \pause
  Device files:
  \begin{itemize}
    \item \texttt{/dev/random}: historically blocking, but not any more ($\ge$ kernel 5.6)
    \item \texttt{/dev/urandom}: historically non-blocking
    \pause
    \item Both are a CSPRNG outputs seeded from the entropy pool
    \pause
    \item Documentation is commonly obsolete
    \pause
    \item Beware of flavour/kernel differences!
  \end{itemize}

  \vspace*{1em}
  
  \pause
  Beware of randomness in VMs!
  \begin{itemize}
    \item A nice \href{https://blogs.oracle.com/linux/post/rngd1}{\emph{blog post}} by Oracle.
  \end{itemize}
\end{frame}

\begin{frame}{Platform functions}
  \begin{itemize}[<+(1)->]
    \item Linux: \href{https://man7.org/linux/man-pages/man2/getrandom.2.html}{\texttt{getrandom()}}
    \begin{itemize}
      \item \texttt{GRND\_RANDOM} flag for \texttt{random} vs \texttt{urandom}
      \item Beware of obsolete documentation (e.g. \texttt{GRND\_NONBLOCK} flag)!
    \end{itemize}
    \vspace*{1em}
    \item macOS: \href{https://developer.apple.com/documentation/security/secrandomcopybytes(_:_:_:)}{\texttt{SecRandomCopyBytes}}
    \vspace*{1em}
    \item Windows: \href{https://learn.microsoft.com/en-us/windows/win32/api/bcrypt/nf-bcrypt-bcryptgenrandom}{\texttt{BCryptGenRandom}}
    \begin{itemize}
      \item \href{https://learn.microsoft.com/en-us/windows/win32/api/wincrypt/nf-wincrypt-cryptgenrandom}{\texttt{CryptGenRandom}} is being deprecated
      \item For .NET, use \href{https://learn.microsoft.com/en-us/dotnet/api/system.security.cryptography.randomnumbergenerator}{\texttt{RandomNumberGenerator}}
      \item \href{https://learn.microsoft.com/en-us/dotnet/api/system.security.cryptography.rngcryptoserviceprovider}{\texttt{RNGCryptoServiceProvider}} is deprecated
    \end{itemize}
  \end{itemize}
\end{frame}

\begin{frame}{PRNG security}
  \pause
  The output of a CSPRNG should be \emph{indistinguishable} from the output of a TRNG by a \emph{computationally bounded} adversary.

  \vspace*{1em}
  \pause
  The \emph{adversary} (or \emph{distinguisher}) should distinguish between the two worlds/games:
    \begin{align*}
      &\begin{game}{\GAME_0}
        &r\gets\mathsf{TRNG}\\
        &\RETURN r
      \end{game}
      &\begin{game}{\GAME_1}
        &s\gets\mathsf{TRNG}\\
        &r\gets\mathsf{CSPRNG}(s)\\
        &\RETURN r
      \end{game}
    \end{align*}

    \vspace*{1em}

    \pause
    If it cannot guess with better success than $\frac{1}{2}$ then the CSPRNG is \enquote{secure}.
\end{frame}

\begin{frame}
  \frametitle{Trusting randomness}

  \begin{itemize}[<+(1)->]
    \item Mix entropy sources
    \begin{itemize}
      \item Integrated \& external (speed boost)
      \item Client \& server-side
      \item Combine with timestamp (helps with bad randomness)
      \item All these have gotchas!
    \end{itemize}
    \item DYOR:
    \begin{itemize}
      \item Is it open source?
      \item Certified?
      \item Reputable vendor?
      \item Track record?
    \end{itemize}
  \end{itemize}

  \vspace*{1em}

  \pause
  Randomness combined with anything\textsuperscript{*} is random.

  \vspace*{0.5em}
  {\scriptsize \textsuperscript{*}without information loss}
\end{frame}

\begin{frame}
  \frametitle{XOR}

  \pause
  \begin{itemize}[<+->]
    \item XOR ($\oplus$): exclusive OR
    \vspace*{1em}
    \begin{displaymath}
      \begin{array}{|c c|c|}
      a & b & a \oplus b\\
      \hline
      0 & 0 & 0\\
      0 & 1 & 1\\
      1 & 0 & 1\\
      1 & 1 & 0\\
      \end{array}
    \end{displaymath}
    \vspace*{1em}
    \item reversible without information loss
    \item associative: $a \oplus (b \oplus c) = (a \oplus b) \oplus c$
    \item commutative: $a \oplus b = b \oplus a$
    \item self-inverse: $a \oplus a = 0$
    \item identity element: $a \oplus 0 = a$
  \end{itemize}
\end{frame}

\begin{frame}
  \frametitle{One-time pad (OTP)}

  \begin{itemize}[<+->]
    \item Symmetric, deterministic cryptosystem
    \begin{itemize}
      \item Same key for encryption and decryption
      \item One ciphertext for each message
      \item $\ENC_k(m) = m \oplus k = c$
      \item $\DEC_k(c) = c \oplus k = m$
    \end{itemize}
    \vspace*{1em}
    \item Key $k$ of length $\lambda$ bits
    \begin{itemize}
      \item Must be \enquote{perfectly} random (uniformly random)
      \item $k\getsu\{0,1\}^\lambda$
      \item $\lambda \ge \lvert m\rvert$ i.e. key as long as message 
      \item key must only be used once
    \end{itemize}
    \vspace*{1em}
    \item \textbf{Information theoretically secure}, i.e. perfect secrecy
    \begin{itemize}
      \item All decryptions are valid $\iff$ the intended message cannot be identified
      \item Brute force alone is useless even for an unbounded adversary (infinite resources)
    \end{itemize}
  \end{itemize}
\end{frame}

\begin{frame}
  \frametitle{One-time pad (OTP)}
  \begin{center}
    {\color{red} Catastrophic failure if the 3 conditions are not respected!}
  \end{center}
\end{frame}

\begin{frame}
  \frametitle{OTP limitations}

  \pause
  \begin{itemize}[<+->]
    \item Size issues:
    \begin{itemize}
      \item Key as long as message
      \item How do we design and store the codebook?
    \end{itemize}
    \item Practicality issues: 
    \begin{itemize}
      \item New key for every message
      \item How do we synchronise keys?
      \item How to exchange the codebook?
    \end{itemize}
    \item Security issues:
    \begin{itemize}
      \item Malleable i.e. can be meaningfully altered
    \end{itemize}
  \end{itemize}
\end{frame}

\begin{frame}
  \frametitle{Symmetric cryptosystems}

  A \emph{private-key encryption scheme} is a triple of algorithms:
  \begin{itemize}
    \pause\item $\GEN$ is a randomised \emph{key generation algorithm}: $k\gets\GEN$
    \pause\item $\ENC$ is a (possibly randomised) \emph{encryption algorithm}: $c\gets\ENC_k(m)$
    \pause\item $\DEC$ is a \emph{decryption algorithm}: $m\gets\DEC_k(c)$.
  \end{itemize}

  \vspace*{1em}

  \begin{itemize}
    \pause\item $k$ is the \emph{private key}, also called \emph{secret key}
    \pause\item The message $m$ is the \emph{plaintext}
    \pause\item The encrypted message $c$ is the \emph{ciphertext}
  \end{itemize}

  \vspace*{1em}

  \pause
  The cryptosystem is \emph{functional} (achieves \emph{correctness}) if $\DEC_k(\ENC_k(m))=m$.
\end{frame}

\begin{frame}
  \frametitle{Symmetric cryptosystems}
  \framesubtitle{For mathematical rigour}

  Keys, messages, and ciphertexts must belong to \emph{spaces}, i.e. sets of elements:
  \begin{itemize}
    \pause\item Key space $\KKK$
    \pause\item Message space $\MMM$
    \pause\item Ciphertext space $\CCC$
  \end{itemize}

  \vspace*{1em}

  \pause
  The functions are therefore maps
  \begin{itemize}
    \pause\item $\ENC_k:\MMM\to\CCC$
    \pause\item $\DEC_k:\CCC\to\MMM\cup\{\bot\}$
  \end{itemize}

  \vspace*{1em}

  \pause
  The symbol $\bot$ (\enquote{bottom}) denotes decryption failure, i.e. the \emph{abort} symbol. 
\end{frame}

\begin{frame}{Symmetric cryptosystems}
  \framesubtitle{EAV-security}
  \begin{center}
  \begin{tikzpicture}
    \node (alice) at (0,0)      {\includegraphics[height=80px]{alice}};
    \node (bob)   at (8, 0)     {\includegraphics[height=80px]{bob}};
    \node (eve)   at (4, -2.1)  {\includegraphics[height=80px]{eve}};

    \draw[->,thick] (alice.east) -- (bob.west);
    \draw[->,thick] (4, 0) -- (4, -0.5);
    \node[anchor=center,draw,circle,fill=white,thick,minimum size = 2px, inner sep=0pt] at (4,0) {};
    
    \node (c) at (4, 0.2) {$c$};
    \node (gen)   at (4, 2) {$\GEN$};

    \node (ska) at (alice.north) [xshift=-7px,yshift=7px] {$k$};
    \node (skb) at (bob.north) [xshift=7px,yshift=7px] {$k$};

    \draw[<-,dashed,thick] (alice.north) |- (gen.west);
    \draw[->,dashed,thick] (gen.east) -| (bob.north);

    \node[anchor=west] (m) at (-3, 0.25) {$m\gets\MMM$};
    \node[anchor=west] (enc) at (-3, -0.25) {$c\gets\ENC_k(m)$};
    \node[anchor=east] (dec) at (11, 0) {$m\gets\DEC_k(c)$};
  \end{tikzpicture}
  \end{center}

  \pause
  Can Eve guess the message that Alice sent Bob?
\end{frame}

\begin{frame}
  \frametitle{IND-OT-CPA}

  \begin{itemize}[<+->]
    \item Cryptographers speak of \emph{semantic security}
    \item Semantic security is difficult to work with
    \item IND-CPA $\implies$ IND-SEM
    \item We aim for \emph{indistinguishability} instead
    \pause
    \begin{align*}
      &\begin{game}{\GAME_0^{\AD}}
        &k\gets\GEN\\
        &(m_0, m_1)\gets\AD\\
        &\IF \lvert m_0\rvert \neq \lvert m_1\rvert\  \RETURN 0\\
        &\RETURN \ENC_k(m_0)
      \end{game}
      &\begin{game}{\GAME_1^{\AD}}
        &k\gets\GEN\\
        &(m_0, m_1)\gets\AD\\
        &\IF \lvert m_0\rvert \neq \lvert m_1\rvert\  \RETURN 0\\
        &\RETURN \ENC_k(m_1)
      \end{game}
    \end{align*}
  \end{itemize}

  \pause
  The \emph{advantage} of the $t$-time (poly time) adversary should be \emph{negligible}:
  \pause
  \[
    \mathsf{Adv}_{\AD}^{\mathsf{IND\text{-}OT\text{-}CPA}} = 
    \Bigl\lvert
      \Pr{\bigl[\GAME_0 = 1\bigr]} - \Pr{\bigl[\GAME_1 = 1\bigr]}
    \Bigr\rvert \le \varepsilon
  \]
  \pause
  The scheme is then $(t,\varepsilon)$-IND-OT-CPA secure.

\end{frame}

\begin{frame}{Stream ciphers}
  \pause
  \begin{itemize}[<+->]
    \item Follows from the OTP
    \item Symmetric cipher
    \item Combine message with PRNG stream: the \emph{keystream}
    \item Typically bit-by-bit using XOR
    \item The key is the PRNG \emph{seed}
  \end{itemize}

  \vspace*{1em}

  \pause
  For security:
  \pause
  \begin{itemize}[<+->]
    \item Generate a high quality seed
    \item Use a reputable CSPRNG
    \item Do not re-use a keystream
    \item Do not re-use a key
    \item Use a message authentication mechanism (for integrity)
  \end{itemize}
\end{frame}

\begin{frame}{Stream ciphers}
  \begin{itemize}[<+->]
    \item How to decrypt?
    \begin{itemize}
      \item Recipient re-generates the keystream from the seed
    \end{itemize}
    \item How to send multiple messages?
    \begin{itemize}
      \item Goal: keep the key, but use a new stream
      \item Use an \emph{initialisation vector} (IV) sometimes called \emph{nonce}
      \item Update the IV after every encryption
      \item Initialise the PRNG with key and IV (complex \enquote{mixing})
      \item Send the IV to the recipient
    \end{itemize}
    \item IVs are not supposed to be secret: only the key should be secret
    \item Not all stream ciphers use an IV
  \end{itemize}
\end{frame}

\begin{frame}{Stream ciphers}
  Some well-known ciphers include
  \pause
  \begin{itemize}[<+->]
    \item RC4: belongs in a museum, weak!
    \item A5/1: used for GSM, weak!
    \item E0: used for Bluetooth, weak!
    \item Ascon \& ACORN: relatively new, interesting for constrained envs
    \item ChaCha20: used by many \texttt{arc4random} PRNGs, popular
  \end{itemize}

  \vspace*{1em}

  \pause
  Note: NIST does not standardise stream ciphers. Not entirely sure why.

  \vspace*{1em}

  \pause
  Some block ciphers can be turned into stream ciphers through \emph{modes of operation} e.g. AES-CTR.

  \vspace*{1em}

  \pause
  To also provide authenticity: ChaCha20-Poly1305, AES-GCM.
\end{frame}

\end{document}
