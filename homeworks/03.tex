\documentclass{homework}

\title{03}
\date{\DTMdate{2024-11-15}}

\usepackage{crysymb}
\usepackage{fancyvrb}
\fvset{listparameters=\setlength{\topsep}{0pt}\setlength{\partopsep}{0pt}}

\begin{document}
\maketitle

\textbf{Reminders}

\begin{itemize}
  \item Include in your report approximately how long the homework took for you.
  \item Read through task descriptions carefully.
  \item You may request deadline extensions until 12.11.24 23:59 EET.
  \item Deadlines are strict: if you do not submit your assignment by the deadline, you fail it.
  \item To pass the course you must pass all homework assignments.
  \item Second chances:
  \begin{itemize}
    \item If I review your assignment before the deadline, I will \emph{always} give you a chance to fix it for the deadline (or with a slight extension).
    Start early!
    \item If I review your assignment after the deadline, I will give you an extension \emph{only} if you have explained what part of your code does not work, and what you did to attempt to fix it.
    Ask questions when you are stuck to avoid risking it!
    \item If I give you an extension, you must still fix your homework by the extended deadline to pass.
    \item If your code does not work but you are not aware of it or have not explained your issues, you fail the homework (if I review it after the deadline).
    Test your code!
    \item If you fail your homework, you can appeal my decision within 3 days of receiving your grade (grade day + 3).
    Write me an email or message me on Discord.
    I will then organise an oral defence for you.
  \end{itemize}
  \item You may share hints with each-other, but no answers/code.
  \item You may ask me questions (I prefer Discord to Moodle/email).
  If your code does not work and you have not asked me any questions, you will have a hard time arguing that you actually tried solving it!
  \item If I catch you plagiarising/copying, I will report you to the faculty and fail you for the course.
\end{itemize}

\newpage

\begin{center}
  Theory tasks
\end{center}

\textbf{Additional instructions}

\begin{itemize}
  \item Submit as a PDF file called \texttt{studentid-03.pdf} where \enquote{studentid} is your 6 letter student ID/username (\enquote{takraa} for me).
  \item Submit it on Moodle as a standalone file (not as part of the code ZIP).
\end{itemize}

\begin{task}{GPG}
  We briefly covered the web-of-trust approach and the existence of PGP, however, we did not go in depth on what PGP is.

  PGP stands for \emph{Pretty Good Privacy} and is a cryptographic toolset which can be used for encryption and signing of data.
  However, PGP is proprietary software (its history is quite complex) which therefore limits its use by the open source community and the general public.

  Since the functionality provided by the PGP was useful however, eventually the \emph{OpenPGP standard} was developed (\href{https://datatracker.ietf.org/doc/html/rfc4880}{RFC 4880}).
  In July of this year, \href{https://datatracker.ietf.org/doc/html/rfc9580}{RFC 9580} was published to supersede RFC 4880, but the state of implementations is muddy.
  If you are interested in tech drama, see:
  \begin{itemize}
    \item \href{https://lwn.net/Articles/953797/}{A schism in the OpenPGP world},
    \item \href{https://blog.pgpkeys.eu/critique-critique.html}{A Critique on \enquote{A Critique on the OpenPGP Updates}}.
  \end{itemize}

  A complete and free implementation of the OpenPGP standard is the \emph{GNU Privacy Guard}, abbreviated as GPG (or GnuPG, or GPG2).
  GPG has many uses, but the most \enquote{mainstream}\footnote{This is from my own experience, and is likely not a global truth.} are email encryption and signing\footnote{A nice resource: \url{https://emailselfdefense.fsf.org/en/}}, SSH access (SSH can work with GPG, but should it?), and signing git commits\footnotemark{}.
  \footnotetext{\url{https://docs.github.com/en/authentication/managing-commit-signature-verification/signing-commits}}

  \begin{tcolorbox}
    I believe, for various reasons, e.g. the lack of forward secrecy and allowing users and lack of sensible safeguards\footnote{It is particularly easy to do stupid things with PGP due to the abundance of settings.}, that GPG (and PGP) is legacy tooling and should be deprecated for most purposes.
    While not all of the community would agree with me, I recommend that you avoid PGP unless there are no better alternatives for a specific purpose.
  \end{tcolorbox}

  \begin{tcolorbox}
    Git commits can nowadays also be signed with SSH keys.
    GitHub, GitLab and BitBucket all support it.
    SSH signing was added to Git version 2.34.0 in 2021.
    \href{https://github.com/git/git/blob/master/Documentation/RelNotes/2.34.0.txt}{(changelog)}
  \end{tcolorbox}

  GPG is a command-line tool, however different frontends exist\footnote{\url{https://gnupg.org/software/frontends.html}}.
  The GUI which I have seen used the most is Kleopatra, which is available on Windows, macOS and Linux.
  For Windows, the Kleopatra GUI comes with \href{https://gpg4win.org/index.html}{Gpg4win}.
  Still, I typically use the CLI instead of a GUI (in the rare cases when I have to use GPG).

  For this task, read through the following two articles:
  \begin{itemize}
    \item \href{https://www.digitalocean.com/community/tutorials/how-to-use-gpg-to-encrypt-and-sign-messages}{\textit{How To Use GPG to Encrypt and Sign Messages}} (DigitalOcean)
    \begin{itemize}
      \item Pay special attention to the \enquote{Create a Revocation Certificate} section!
      \item Summarise in a few sentences how public keys are typically shared and made available to other parties.
      
      If you make your key available using this method, can you later \enquote{revoke} the availability?
      (Requires a bit of online searching. Include at least two references!)
    \end{itemize}
    \item \href{https://www.gnupg.org/gph/en/manual/x135.html}{\textit{Making and verifying signatures}} (The GNU Privacy Handbook)
    \begin{itemize}
      \item For each of the different signature methods, come up with a use-case where that specific method is suitable.
    \end{itemize}
  \end{itemize}

  Both articles will help you with the practical tasks.

  \textit{Addendum.}
  If you have to use GPG and want to use it with a YubiKey, check out the article \href{https://support.yubico.com/hc/en-us/articles/360013790259-Using-Your-YubiKey-with-OpenPGP}{\textit{Using Your YubiKey with OpenPGP}}.
\end{task}

\begin{task}{Certificate ToS}
  Read through the short summary on \href{https://www.id.ee/en/article/compete-information-about-the-terms-of-use-of-certificates/\#what-you-need_2}{\textit{What you need to know about the terms and conditions for the use of ID-card certificates}}.

  Then, answer the following questions:
  \begin{itemize}
    \item What must you do when you lose your document or it is stolen?
    \item How can you prevent the unauthorised use of your document?
    \item What is the difference between suspending certificates and declaring them invalid?
    \item Can you anonymously use your document digitally?
    \item Does the ID card serve any purpose after its certificates have been invalidated?
  \end{itemize}
\end{task}

\begin{task}{ASiC-E containers}
  Task file: \texttt{pubkey.asice}

  Investigate the signature container, then answer the following questions:
  \begin{itemize}
    \item What was the time on my computer at the moment of singing (Estonian local time)?
    \item What is the signing time reported by DigiDoc (Estonian local time)?
    \item Do these times differ? Why yes/why not?
    \item Did I issue the signature with my ID-card or Smart-ID?
    How do you know which?
  \end{itemize}
\end{task}

\newpage
\setcounter{task}{0}

\begin{center}
  Practical tasks
\end{center}

\textbf{Additional instructions}

\begin{itemize}
  \item I must be able to run your program with GnuPG v2.4.5 and OpenSSL v3.3.2\footnotemark{}.
  \footnotetext{You may use older versions of GnuPG and OpenSSL as long as you do not use deprecated functionality.}
  \item You can use whichever tools or programming language to solve the following tasks.
  However, you must provide code comments to explain what you are doing and why, and reference all used tools in your PDF.
  \item ZIP all files (except for the report) in a zip called \texttt{data.zip}.
  \item Your ZIP must have at least the following structure/contents:
  \begin{Verbatim}
data/
├── revocation.cdoc
├── studentid.asice
├── spurious.asice
├── ed25519.pub
└── message.sig
  \end{Verbatim}
  where \enquote{studentid} is your student identifier (\enquote{takraa} for me).
  Add your additional code files to the ZIP as well, if any.
\end{itemize}

\begin{task}{PGP keys}
  Generate a PGP primary key that can only certify other keys, but not sign or encrypt, and which is valid for two months.
  Then, generate a subkey for it which can only sign data, but not encrypt, and which is valid for one month.
  Please do not publish your key to any keyserver online!

  Both keys must use Ed25519.
  For the key metadata, use your real name and your TalTech email.
  For the comment, use \enquote{Homework 3}.
  \autoref{fig:gpg} displays what your key should look like on the command line.
  You can also find my PGP public key in \texttt{pubkey.asice}.
  Pay attention to the \texttt{[C]} and \texttt{[S]} markers.

  \begin{figure}[h!]
    \center
    \includegraphics[width=\textwidth]{../images/gpg_pubkey}
    \caption{Example metadata for a key with the required parameters.}
    \label{fig:gpg}
  \end{figure}

  Include a screenshot showing that the key is in your keyring in your PDF report (CLI and GUI are both fine).

  Then, generate a revocation certificate for the primary key.

  \begin{tcolorbox}
    If you ever have to generate a PGP key, make sure to always also generate a revocation certificate!
    Make sure to store this certificate in a safe but available (for you) manner.
    This allows you to revoke a key should you lose it or should it become compromised.
    \tcblower
    Proper key management with GPG is quite complex, e.g. due to the subkey system and the usual complexity of rotating keys.
    As such, revocation certificates become especially important for potential damage control.
  \end{tcolorbox}

  Encrypt the revocation certificate to my ID code using DigiDoc.
  You can get my ID code from the signed container.
  \begin{tcolorbox}
    In practice, you should of course never leak your revocation certificate, as anyone can permanently revoke your key with it.
  \end{tcolorbox}

  Add the file as \texttt{revocation.cdoc} into the submission ZIP.
\end{task}

\begin{task}{Propagating trust}
  Task file: \texttt{pubkey.asice}

  Export your PGP public key from the previous task into a file in PEM format and sign it with an Estonian digital signature (ID-card, Mobile-ID, Smart-ID, digi-ID, \dots) with DigiDoc.
  
  Upload the file as \texttt{studentid.asice} into the submission ZIP, where \enquote{studentid} is your student identifier (\enquote{takraa} for me).
  \begin{tcolorbox}
    If you are living in Estonia, you are required by law to have either an Estonian ID card, an Estonian ID card for a EU citizen, or a residence permit card.
    If you do not have one, I strongly suggest applying for one.
    \tcblower
    If you do not have such a document\footnote{I will not report you, do not worry. Also, do encrypt your message to my ID code using DigiDoc.}, or do not have any currently valid certificates, please write me an email.
  \end{tcolorbox}

  Then, import my public key from the \texttt{pubkey.asice} container into your keyring.
  Assign it a trust level of 4, i.e. \enquote{I trust fully}.

  Include a screenshot showing that the key is in your keyring with the appropriate trust level in your PDF report (CLI and GUI are both fine).
\end{task}

\newpage

\begin{task}{Are we the bad guys?}
  Task file: \texttt{pubkey.asice}

  Your task is to modify the signature container in such a way that DigiDoc4 displays the string \enquote{Signature is not valid} (\autoref{fig:digidoc}).
  Explain what you did to make it so and include commands/screenshots.

  \begin{figure}[h!]
    \center
    \includegraphics[width=\textwidth]{../images/invalid_sig}
    \caption{Expected signature status displayed by DigiDoc.}
    \label{fig:digidoc}
  \end{figure}

  Upload the file as \texttt{spurious.asice} in the homework submission on Moodle.

  \textit{Hint.}
  The \href{https://www.id.ee/wp-content/uploads/2021/06/bdoc-spec212-eng.pdf}{BDOC 2.1} reference can help you if you are unable to re-assemble the container in a way such that DigiDoc actually opens it.
  You can also compare the \href{https://man.freebsd.org/cgi/man.cgi?query=zipinfo}{\texttt{zipinfo}} output of the original container with the container you have assembled if you are having issues.
\end{task}

\begin{task}{\$\%@!}
  Generate an Ed25519 signing key with OpenSSL.
  Export the public key into a file called \texttt{ed25519.pub}.
  Then, use it to sign the file \texttt{message.txt} and save the signature as \texttt{message.sig}.
  Include the generation and signing commands in your report, and the two files in the submission ZIP.
\end{task}

\newpage

\begin{task}{A bit of everything}
  Homework challenge file: \texttt{challenges.txt}

  In the homework challenge file, you will find two columns of data: the first column includes salted hashes of student usernames (e.g. takraa) while the second column includes SHA-256 hashes of registration codes of Estonian organisations.
  Each student is assigned a different organisation.

  Your task is then to:
  \begin{enumerate}
    \item Find which entry corresponds to you (i.e. hashes your student username).
    \item Find the company your challenge is for (i.e. find the SHA-256 pre-image).
    \item Find the certificate serial number of any e-Seal certificate of that company.
    \begin{itemize}
      \item Certificates are issued by SK ID Solutions.
      \item I say \enquote{any}, since an organisation might have multiple e-Seal certificates.
      \item NB! Not all certificates are for e-Seals (e.g. some are for encryption).
      \item Do not confuse the certificate serial number with the subject serial (the subject serial is likely the same as the registry code).
      \item The certificate serial has at least 20 decimal digits (in our case).
    \end{itemize}
  \end{enumerate}

  Include in your report:
  \begin{itemize}
    \item The registration code of the company/organisation.
    \item The name of the company.
    \item The certificate serial as a base10 integer.
  \end{itemize}

  Include any code you used to solve the task in the submission ZIP.
  Provide commands and/or screenshots for any steps you performed manually, i.e. that was not solved by your code (if any).

  Registration codes in Estonia are 8 decimal digits long.
  The following should help you narrow the search space:
  \begin{itemize}
    \item Registry codes of companies (AS/OÜ) begin with a \enquote{1}.
    \item Registry codes of public institutions begin with a \enquote{7}.
    \item Registry codes of nonprofit organisations (MTÜ) begin with an \enquote{8}.
    \item Registry codes of foundations (sihtasutus) begin with a \enquote{9}.
  \end{itemize}
\end{task}
\end{document}
