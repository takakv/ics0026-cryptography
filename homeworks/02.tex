\documentclass{homework}

\title{02}
\author{Taaniel Kraavi}
\date{\DTMdate{2024-03-21}}

%\usepackage{etoolbox}
%\makeatletter
%\preto{\@verbatim}{\topsep=0pt \partopsep=-2\parsep }
%\makeatother

\usepackage{crysymb}

\usepackage{fancyvrb}
\fvset{listparameters=\setlength{\topsep}{0pt}\setlength{\partopsep}{0pt}}

\usepackage{listings}

\lstset{
basicstyle=\small\ttfamily,
columns=flexible,
breaklines=true
}

\begin{document}
\maketitle

\textbf{Reminders}

\begin{itemize}
  \item Include in your report approximately how long the homework took for you.
  \item If you want feedback on Discord, include your username in your report.
  \item Read through task descriptions carefully.
  \item You may request deadline extensions until 17.03.24 23:59 EET.
  \item Deadlines are strict: if you do not submit your assignment by the deadline, you fail it.
  \item To pass the course you must pass all homework assignments.
  \item Second chances:
  \begin{itemize}
    \item If I review your assignment before the deadline, I will \emph{always} give you a chance to fix it for the deadline (or with a slight extension).
    Start early!
    \item If I review your assignment after the deadline, I will give you an extension \emph{only} if you have explained what part of your code does not work, and what you did to attempt to fix it.
    Ask questions when you are stuck to avoid risking it!
    \item If I give you an extension, you must still fix your homework by the extended deadline to pass.
    \item If your code does not work but you are not aware of it or have not explained your issues, you fail the homework (if I review it after the deadline).
    Test your code!
    \item If you fail your homework, you can appeal my decision within 3 days of receiving your grade (grade day + 3).
    Write me an email or message me on Discord.
    I will then organise an oral defence for you.
  \end{itemize}
  \item You may share hints with each-other, but no answers/code.
  \item You may ask me questions (I prefer Discord to Moodle/email).
  If your code does not work and you have not asked me any questions, you will have a hard time arguing that you actually tried solving it!
  \item If I catch you plagiarising/copying, I will report you to the faculty and fail you for the course.
\end{itemize}

\newpage

\begin{center}
  Theory tasks
\end{center}

\textbf{Additional instructions}

\begin{itemize}
  \item Submit as a PDF file called \texttt{studentid-02.pdf} where `studentid' is your 6 letter student ID/username (`takraa') for me.
  \item Submit it on Moodle as a standalone file (not as part of the code ZIP).
\end{itemize}

\begin{task}{ECIES}
  The Elliptic Curve Integrated Encryption Scheme (ECIES) is a hybrid cryptosystem framework: it combines a \emph{key encapsulation mechanism} (KEM) with a \emph{data encapsulation mechanism} (DEM) that is also \emph{authenticated}.

  As a rough overview, the following can be parametrised:
  \begin{itemize}
    \item the key derivation function (KDF), e.g. KDF2;
    \item the hash function, e.g. SHA-512;
    \item the scheme for authenticating the encryption, e.g. HMAC-SHA-512;
    \item the symmetric encryption scheme, e.g. AES-256-CBC;
  \end{itemize}
  on top of different elliptic curve domain parameters.
  Some implementations even allow to change the key-agreement protocol (variations of DH).
  Combinations are also possible.
  For example, when using XChaCha20-Poly1305, a separate message-authentication scheme is no longer needed.

  Standards and implementations limit the allowed values for each of these parameters, which helps against misuse (e.g. picking weak options) but complicates interoperability.
  For example, no implementation can comply with all ECIES-defining standards.
  Furthermore, not all options and/or combinations provide equivalent security.
  Picking `strong' parameters while maintaining interoperability is therefore quite tricky. 

  For this task, find three standards where ECIES is defined.
  Then, highlight (preferably in a table) which parameters/algorithms are supported by all three standards (if any).
\end{task}

\begin{task}{Password switcheroo}
  Oh no, you mistakenly encrypted your newly generated secret key with a really weak password (or even worse, you did not password protect it).
  Write down the OpenSSL command that you would use to change the password of your asymmetric secret key.

  If you want to check whether your command works, I have uploaded to Moodle the encrypted RSA private key file \texttt{demokey.pem} that is encrypted with the password \texttt{changeme}.
\end{task}

\begin{task}{ROCA}
  Read \href{https://crocs.fi.muni.cz/public/papers/rsa_ccs17}{\emph{this blog post}} by the researchers who discovered ROCA.
  More than 750 000 Estonian ID cards were impacted by this vulnerability.

  Summarise the discovery (what, why, how to fix) in no more than five sentences.
  Then succinctly answer the following:
  \begin{itemize}
    \item Did the researchers completely break 2048-bit RSA?
    \item Are ECC keys generated on the affected chips vulnerable?
    \item How can you detect if your keys are vulnerable?
    \item What is the worst-case price for factorising a 2048-bit impacted key?
    \item Name three mitigation strategies if your hardware is impacted.
  \end{itemize}
\end{task}

\begin{task}{Common attacks against RSA}
  Do some digging, then name a technique (with a source) for cracking RSA when:
  \begin{itemize}
    \item $p$ and $q$ are very close,
    \item the modulus $n$ is re-used with different private keys,
    \item the private exponent is small,
    \item the public exponent is small and OAEP is not used.
  \end{itemize}

  Give three examples why OAEP padding is important.
\end{task}

\begin{task}{Leaky ElGamal}
  Recall that for some secret $x\getsu\ZZ_q$ and generator $g$, the ElGamal public key is $\PK\gets g^x$.
  Encryption is then carried out with:
  \[
    r \getsu \ZZ_q, \ENC_\PK(m; r) = \bigl(g^r, m \cdot \PK^r\bigr) = (u, v) = c
  \]
  where $r$ is the randomness which makes ElGamal a probabilistic encryption scheme.

  Suppose now that $r$ is leaked meaning that you know the $r$ used during the encryption.
  Are you then able to recover $m$ from $c$ without the secret key?
  If so, how to do it?

  Can you also decrypt other messages (without gaining access to new values of $r$)?
\end{task}

\newpage
\setcounter{task}{0}

\begin{center}
  Practical tasks
\end{center}

\textbf{Additional instructions}

\begin{itemize}
  \item I must be able to run your program with Python 3.12 and OpenSSL v3.2.1\footnotemark{}.
  \footnotetext{You may use older versions of Python and OpenSSL as long as you do not use deprecated functionality.}
  \item Do not import any 3\textsuperscript{rd} party module\footnotemark{} not specified in the template files.
  \footnotetext{You may change \texttt{Crypto} to \texttt{Cryptodome} if needed.}
  You are free to import any built-in module.
  If you feel the need to import some other 3\textsuperscript{rd} party module, please check with me beforehand!
  \item Your program must not have any unhandled errors (print a message and exit with an error code without printing the trace) except for file-based errors, e.g. file read/write errors.
  \item Submit on Moodle as a ZIP file called \texttt{code.zip}.
  \item Your Python files must have the same name as the template files but without the \texttt{template} suffix, e.g. \texttt{interop\_template.py} becomes \texttt{interop.py}.
  \item You may change everything in the template files as long as I can run your program with the CLI arguments specified in the templates.
  \item Your ZIP must have the following structure/contents:
  \begin{Verbatim}
code/
├── interop.py
└── homomorphic.py
  \end{Verbatim}
\begin{Verbatim}
  
\end{Verbatim}
\end{itemize}

\begin{task}{RSA interop}
  Homework template file: \texttt{interop\_template.py}

  Create a Python3 program that performs RSA-OAEP\footnote{\url{https://en.wikipedia.org/wiki/Optimal_asymmetric_encryption_padding}} encryption/decryption.
  Recall from the lecture that OAEP padding is required for probabilistic encryption, and to prevent some subtle attacks.

  Your program must provide three separate tools:
  \begin{itemize}
    \item \texttt{genpkey} --- generate an RSA private key

    Depending on the parameter, your program must generate a $3072$-bit RSA private key either with OpenSSL or with PyCryptodome and export it into a file in PEM format.

    When generating a private key with OpenSSL, your program \emph{must not} print the key-generation progress bar to the console.

    The private key must be password protected, and encrypted with AES-256 in CBC mode in the PKCS\#8 format.
    \begin{itemize}
      \item For PyCryptodome, you should use \texttt{PBKDF2WithHMAC-SHA512AndAES256-CBC} with an iteration count of $210 000$ when setting up the private key encryption.
      Why these parameters?
      They are in the documentation and are fairly sensible.
      \item New versions of OpenSSL (> 3.0) should use PKCS\#8 by default if you export an encrypted private key in PEM, but do not allow for such a fine-grained control with \texttt{genpkey} as PyCryptodome.
      
      If you want more control, you can check out \href{https://www.openssl.org/docs/man3.2/man1/openssl-pkcs8.html}{\texttt{openssl-pkcs8}}.
    \end{itemize}

    If you are unsure whether your exported private key is in PKCS\#8 format and do not want to read through \href{https://datatracker.ietf.org/doc/html/rfc5208}{RFC 5208}, you can simply dump the ASN.1 structure of both your private key and of the example private key given on Moodle.
    If both seem to have a similar structure, your key is using PKCS\#8.

    Example invocation:
    \begin{Verbatim}
python3 interop.py genpkey --openssl \
    --out key.pem --pass secret
    \end{Verbatim}

    \item \texttt{pkey} --- file encryption
    
    Your program must extract the public key from a private key with either PyCryptodome or with OpenSSL depending on the parameter.

    Example invocation:
    \begin{Verbatim}
python3 interop.py pkey --openssl \
    --in key.pem --out pub.pem --pass secret
    \end{Verbatim}

    \item \texttt{pkeyutl} --- file decryption
    
    Your program must read either the private or public key from the specified file, a plaintext or ciphertext from the specified file, and output the encryption or decryption to the specified file.

    Depending on the parameter, your program must use either PyCryptodome or OpenSSL for the encryption/decryption.

    Example invocation for encryption:
    \begin{Verbatim}
python3 interop.py pkeyutl --openssl --encrypt \
    --inkey pub.pem --in message.txt --out data.enc
    \end{Verbatim}

    Example invocation for decryption:
    \begin{Verbatim}
python3 interop.py pkeyutl --openssl --decrypt \
    --inkey key.pem --in data.enc --out data.enc \
    --pass secret
    \end{Verbatim}
  \end{itemize}

  Make sure to handle errors by printing an informational message and exiting (\texttt{sys.exit()}) with an error code without printing the exception trace.
  Your program must not crash if missing parameters are provided, the wrong password is provided, OpenSSL fails, etc.
  Some error handling and hints are part of the template, but not all of it.

  You do not however have to (but you may) check for file I/O errors, i.e. you may assume that if you provide a filename, then that filename does exist or can be written to.

  The password provided on the command line must be treated as an UTF-8 string.
  Remember to wrap the password in quotes if you want to include spaces!

  \begin{tcolorbox}[title=Note]
    In practice, providing secrets (e.g. keys, passwords, etc.) directly on the CLI as an argument to a program is a terrible idea, since they typically get logged into your terminal history!
  \end{tcolorbox}

  You can check whether your code is functional with the script \texttt{interop\_check.py}.
  This script does not test for error handling, so do not consider it as the definitive validator.
  I have not included the error handling checks in this script since it would give away the edge cases and I want you to think about them yourself.
\end{task}

\begin{task}{256-bit RSA}
  The following data represents a 256-bit RSA public key.
  \begin{Verbatim}
-----BEGIN PUBLIC KEY-----
MDwwDQYJKoZIhvcNAQEBBQADKwAwKAIhAITRpRf+PNWruknCVkz+bCvD+09P84Gf
CBSO7oyM87qHAgMBAAE=
-----END PUBLIC KEY-----
  \end{Verbatim}
  Find the two RSA primes used for obtaining this key and include them in your report.
  Show how you obtained the primes (preferably with a screenshot of reasonable resolution) and include any applicable links.
\end{task}

\begin{task}{Homomorphism and semiprimes}
  Homework template file: \texttt{homomorphic\_template.py}

  The template contains a hard-coded ElGamal public key, an ElGamal encryption function (\texttt{pk.encrypt()}) and a ciphertext product function (\texttt{pk.multiply()}).
  Recall that an ElGamal ciphertext $c$ is comprised of two parts: $c = (u, v) = \bigl(g^r, m\cdot\PK^r\bigr)$.

  Your program must accept an ElGamal ciphertext as two integers on the command line (see the parsed arguments in the template).
  Your program must produce a ciphertext that decrypts to the same value as the challenge ciphertext.
  However, your ciphertext may not be the same as the provided challenge ciphertext, i.e. $(u, v)$ must be different.

  Your program should print this ciphertext: $u$ as a base10 integer and $v$ as a base10 integer on the next line.
  E.g.
  \begin{Verbatim}
python3 homomorphic.py 157928...613543 296738...721947
1058509285116...9238765983746
6819523011802...1625917583932
  \end{Verbatim}
  without the dots in the middle.
  
  Your task succeeds if:
  \begin{itemize}
    \item $\DEC_\SK\bigl((u_\mathit{challenge}, v_\mathit{challenge})\bigr) = \DEC_\SK\bigl((u_\mathit{you}, v_\mathit{you})\bigr)$, and
    \item $(u_\mathit{challenge}, v_\mathit{challenge}) \neq (u_\mathit{you}, v_\mathit{you})$.
  \end{itemize}
\end{task}

\end{document}
