\title{ICS0026 Cryptography}
\subtitle{Hardware security \& side channels}
\date{\today}
\author{Taaniel Kraavi}
\institute%
{%
  \textit{IT College}\\
  \textit{Tallinn University of Technology}
}

\begin{document}
\begin{frame}
  \titlepage
\end{frame}

\begin{frame}{A need for physical security}
  Access is the bane of security
  \begin{itemize}[<+(1)->]
    \item Client-side tampering
    \item Malicious access
    \item Data exfiltration / leakage
    \item \dots
  \end{itemize}

  \vspace*{1em}

  \pause
  Physical measures provide some tamper resistance.
  \begin{itemize}[<+(1)->]
    \item Trusted components in a BYOD world
    \item Trusted hardware $\neq$ trusted perimeter
    \item Not invulnerable to attack
  \end{itemize}
\end{frame}

\begin{frame}{Secure cryptoprocessors}
  Processors for cryptographic operations with physical security measures.
  \begin{itemize}[<+(1)->]
    \item Hardware security modules (HSM)
    \item Trusted platform modules (TPM)
    \item Smart cards (e.g. bank card, EstEID)
    \item Security tokens (e.g. Yubikey)
  \end{itemize}

  \vspace*{1em}

  \pause
  Varying degree of physical security.
  \begin{itemize}[<+(1)->]
    \item E.g. smartcards typically more vulnerable than HSMs
    \item Uneven difficulty of reverse engineering/probing the chip
    \item What attacks does the device protect against?
  \end{itemize}
\end{frame}

\begin{frame}{Trusted Platform Module (TPM)}
  \pause
  The Trusted Platform Module (TPM) is a standard\dots
  \begin{itemize}[<+(1)->]
    \item Maintained \& published by the Trusted Computing Group (TCG)
    \item TPM Main Specification Version 1.2 (2009)
    \item TPM Library Specification Version 2.0 (2014)
    \begin{itemize}
      \item Latest version available \href{https://trustedcomputinggroup.org/resource/tpm-library-specification/}{\textit{here}}
      \item \enquote{Library approach} for flexibility
    \end{itemize}
    \item Official standard: ISO/IEC 11889:2015
  \end{itemize}

  \pause
  \dots{}but chips implementing it are also called TPMs.
  \begin{itemize}
    \item Reference implementation: {\small\url{https://github.com/Microsoft/ms-tpm-20-ref}}
  \end{itemize}
\end{frame}

\begin{frame}{TPM chips}
  TPMs offer hardware-level protection of authentication material.
  \begin{itemize}[<+(1)->]
    \item Discrete, integrated (e.g. chipset), or firmware (e.g. CPU)
    \begin{itemize}
      \item software and virtual TPMs exist\textsuperscript{*}
    \end{itemize}
    \item Secure storage of authentication artefacts
    \begin{itemize}
      \item e.g. cryptographic keys, fingerprints, passwords, \dots
    \end{itemize}
    \item Platform configuration registers (PCRs)
    \begin{itemize}
      \item e.g. hardware configuration, firmware versions, \dots
      \item useful for sealing \& attestation
    \end{itemize}
  \end{itemize}

  \vspace*{1em}

  \pause
  The TPM provides a hardware \enquote{root of trust}.
\end{frame}

\begin{frame}{TPM use cases}
  %Example functionality:
  %\begin{itemize}[<+(1)->]
  %  \item Cryptographic RNG
  %  \item Cryptographic key generation, binding/wrapping, and sealing
  %  \item Remote attestation \& other trusted computing (TC) functionality
  %\end{itemize}

  %\vspace*{1em}

  Example use cases:
  \begin{columns}[T,onlytextwidth]
    \begin{column}{0.5\textwidth}
      \begin{itemize}[<+(1)->]
        \item Platform integrity \& secure boot
        \item Disk encryption
        \item Virtual smart cards\\(e.g. Windows Hello)
      \end{itemize}
    \end{column}
    \begin{column}{0.5\textwidth}
      \begin{itemize}[<+(1)->]
        \item Remote device attestation
        \item Kernel-level anti cheat
        \item Anti-hammering\\(brute force protection)
      \end{itemize}
    \end{column}
  \end{columns}
\end{frame}

\begin{frame}{TPM implementation security}
  \begin{center}
    \href{https://trustedcomputinggroup.org/wp-content/uploads/TPM-2.0-A-Brief-Introduction.pdf}{TCG's introduction to TPM 2.0}
  \end{center}
\end{frame}

\begin{frame}{TPM sniffing attack}
  A discrete TPM must communicate with the CPU.
  \begin{itemize}[<+(1)->]
    \item TPM is tamper resistant
    \item Communication might not be encrypted
    \item Steal secrets by sniffing the bus
    \item Example: \href{https://pulsesecurity.co.nz/articles/TPM-sniffing}{BitLocker key recovery}
  \end{itemize}

  \vspace*{1em}

  \pause
  Much harder to perform on a firmware TPM, although other attacks may exist.
\end{frame}

\begin{frame}{Trusted Execution Environment (TEE)}
  A Trusted Execution Environment is a secure area of a CPU.
  \begin{itemize}[<+(1)->]
    \item Virtual subsystem or powered by a coprocessor
    \item Code and data is protected: confidentiality \& integrity
    \item Encrypted (and isolated) memory regions called \emph{enclaves}
    \item Code runs within these regions
    \item Computation result returned to the calling process
    \item Remote verification can enable TEE in the cloud (predefined hardware attestation)
  \end{itemize}  
\end{frame}

\begin{frame}{TEE hardware support}
  TEE commonly implemented on top of:
  \begin{itemize}[<+(1)->]
    \item AMD Platform Security Processor (PSP) 
    \begin{itemize}
      \item AMD Secure Encrypted Virtualization (SEV)
    \end{itemize}
    \item Intel Management Engine (ME)
    \begin{itemize}
      \item Intel Software Guard Extensions (SGX)
    \end{itemize}
    \item Apple Secure Enclave Processor (SEP)
    \item ARM TrustZone
  \end{itemize}

  \vspace*{1em}
  
  \pause
  NB! These technologies are very different.
\end{frame}

\begin{frame}{Hardware Security Module (HSM)}
  Hardware Security Modules (HSM)
  \begin{itemize}[<+(1)->]
    \item physical devices
    \item attach to computers/servers
    \item safeguard keys and perform cryptographic/confidential operations
    \item may contain multiple cryptoprocessor chips (e.g. specialised + general purpose)
  \end{itemize}
\end{frame}

\begin{frame}{HSMs (cont.)}
  \pause
  Fancier features (e.g. banking, CAs):
  \begin{itemize}[<+(1)->]
    \item tamper resistance \& responsiveness (e.g. sleep or destruction)
    \item high performance (e.g. live certificate signing)
    \item high availability modes (e.g. failover mechanisms)
    \item certified (e.g. Common Criteria, FIPS)
  \end{itemize}
\end{frame}

\begin{frame}{Smart cards \& EMV}
  Smart cards:
  \begin{itemize}[<+(1)->]
    \item Cards with an integrated circuit chip
    \item Can be used for cryptographic operations (e.g. EstEID, SIM cards)
    \item Complex cards may have additional components (e.g. display, buttons)
    \item NFC for contactless: operation still performed by microchip
    \item Java Card technology: \enquote{applets} written in Java
  \end{itemize}

  \vspace*{1em}

  \pause
  EMV:
  \begin{itemize}[<+(1)->]
    \item Europay MasterCard Visa
    \item Smart payment cards \& terminals
    \item Enables cryptographic validations
  \end{itemize}
\end{frame}

\begin{frame}{Common Criteria (CC)}
  Hardware security is often closed source.
  \begin{itemize}[<+(1)->]
    \item Backdoors?
  \end{itemize}

  \vspace*{1em}

  \pause
  Common Criteria (CC) certification:
  \begin{itemize}[<+(1)->]
    \item Certification bodies (e.g. ANSSI, BSI) and testing labs must be approved
    \item Evaluation Assurance Level (EAL): 1--7 + augmentation
    \item Target of Evaluation (TOE): product being evaluated
    \item Security Target (ST): security properties of the TOE
    \item Protection Profile (PP): security requirements for a device class
    \begin{itemize}
      \item Often a template for the ST
    \end{itemize}
  \end{itemize}
\end{frame}

\begin{frame}{Yubikeys}
  Yubikeys are hardware authentication devices.
  \begin{itemize}[<+(1)->]
    \item Not the only ones, but a popular choice
    \item FIDO2, Personal Identity Verification (PIV) and OpenPGP
    \item FIDO2 support: resident \& non-resident keys (e.g. SSH)
  \end{itemize}

  \vspace*{1em}

  \pause
  If you work in security, you should have at least two.
  \begin{itemize}[<+(1)->]
    \item Can be another reputable brand
    \item At least two, with one as a backup (all 2FA is not equal)
  \end{itemize}
\end{frame}

\begin{frame}{Side channels}
  Side channels leaks are leaks of information through the non-primary communication channel, i.e. the \emph{side channel}.
  \begin{itemize}[<+(1)->]
    \item E.g. fat envelope for uni admittance, slim envelope for rejection
    \item Unintentional leak of \enquote{physical} information
    \item Do not confuse with covert channels (intentional)
  \end{itemize}
\end{frame}

\begin{frame}{Side channels (cont.)}
  Side channels are the bane of implementation.
  \begin{itemize}[<+(1)->]
    \item \enquote{unknown unknowns}
    \item Software \& hardware problem
    \item Broken implementations of theoretically secure primitives
    \item Can be considered during theoretical design
    \begin{itemize}
      \item Easier/harder to implement in a side-channel mitigating way
      \item Point out which parts must be side-channel resistant
      \item E.g. \href{https://ed25519.cr.yp.to/ed25519-20110926.pdf}{EdDSA}
    \end{itemize}
  \end{itemize}
\end{frame}

\begin{frame}{Side channel classes}
  \begin{columns}[T,onlytextwidth]
    \begin{column}{0.5\textwidth}
      \pause
      Logical:
      \begin{itemize}[<+(1)->]
        \item Timing analysis
        \item Behavioural patterns (e.g. size)
      \end{itemize}

      \pause
      Processor shenanigans:
      \begin{itemize}[<+(1)->]
        \item Power analysis
        \item Fault analysis/injection (active)
        \item Cache-based (access, timing)
        \item Frequency analysis (\href{https://www.hertzbleed.com}{Hertzbleed})
      \end{itemize}
      \end{column}
    \begin{column}{0.4\textwidth}
      \pause
      Electromagnetic:
      \begin{itemize}[<+(1)->]
        \item Leakage from cables
        \item RF radiation
      \end{itemize}

      \pause
      Environmental:
      \begin{itemize}[<+(1)->]
        \item Acoustic, optical
      \end{itemize}

      \pause
      Social side channels:
      \begin{itemize}[<+(1)->]
        \item Human bias
        \item \enquote{The Pizza Meter}
      \end{itemize}
    \end{column}
  \end{columns}

  \vspace*{2em}

  \pause
  The list is not exhaustive\dots
\end{frame}

\begin{frame}{Examples}
  Some cool \& scary examples\dots
  \begin{itemize}[<+(1)->]
    \item {\small\url{http://people.csail.mit.edu/mrub/VisualMic/}} (and that was a while ago...)
    \item {\small\url{https://youtu.be/BpNP9b3aIfY}}
  \end{itemize}

  \vspace*{1em}

  \pause
  \dots{}and names you should know.
  \begin{itemize}[<+(1)->]
    \item Meltdown and Spectre: {\small\url{https://meltdownattack.com}}
    \item Hertzbleed: {\small\url{https://hertzbleed.com}}
  \end{itemize}
\end{frame}

\begin{frame}{What can be done?}
  Some approaches:
  \begin{itemize}[<+(1)->]
    \item General chip design with side-channels in mind
    \item Dedicated hardware implementations (e.g. some cryptoprocessors)
    \item Electromagnetic and/or acoustic shielding
    \item Randomisation of operations (masking and noise)
    \item Design algorithms with side-channels in mind
    \item Constant time operations
  \end{itemize}

  \vspace*{1em}

  \pause
  Side-channel protection can likely never be absolute.
  \begin{itemize}[<+(1)->]
    \item Protect multiple levels: transistor, program, algorithm, and protocol level
  \end{itemize}
\end{frame}

\begin{frame}{Intel's recommendations}
  Intel offers some guidance:
  \begin{itemize}
    \item \href{https://www.intel.com/content/www/us/en/developer/articles/technical/software-security-guidance/secure-coding/security-best-practices-side-channel-resistance.html}{\textit{Security Best Practices for Side Channel Resistance}}
    \item \href{https://www.intel.com/content/www/us/en/developer/articles/technical/software-security-guidance/secure-coding/mitigate-timing-side-channel-crypto-implementation.html}{\textit{Guidelines for Mitigating Timing Side Channels Against Cryptographic Implementations}}
  \end{itemize}
\end{frame}

\begin{frame}{Library mitigations}
  What do general purpose libraries do?
  \begin{itemize}[<+(1)->]
    \item Libraries cannot usually rely on secure hardware
    \item Attempt to mitigate timing and access issues in software
    \item Some articles/examples:
    \begin{itemize}
    \item \href{https://botan.randombit.net/handbook/side_channels.html}{Botan's side channels handbook}
    \item \href{https://www.bearssl.org/constanttime.html}{Why Constant-Time Crypto? (BearSSL)}
    \item \href{https://cryspen.com/post/ml-kem-implementation/}{Verified ML-KEM (Kyber) in Rust (Cryspen)}
    \end{itemize}
  \end{itemize}
\end{frame}

\begin{frame}{Constant time cryptography}
  \pause
  Execution should not depend on secret data:
  \begin{itemize}[<+(1)->]
    \item No branching on secret data
    \item No memory access based on secret data
    \item No variable time instructions with secret data (e.g. DIV on x86)
  \end{itemize}

  \vspace*{1em}
  
  \pause
  Cryptographer's will $\neq$ compiler's will.
  \begin{itemize}[<+(1)->]
    \item The compiler may optimise away constant time algorithmic
    \item Constant time languages/compilers attempted
    \item Speculative constant time? (Spectre)
  \end{itemize}
\end{frame}

\begin{frame}{Issue of unknowns}
  Constant time crypto is easier for fixed parameters.
  \begin{itemize}[<+(1)->]
    \item Constant-time optimised operations for known values (e.g. fixed finite fields)
  \end{itemize}

  \vspace*{1em}

  \pause
  What if parameters are variable?
  \begin{itemize}[<+(1)->]
    \item New RSA modulus for each keypair
    \item Constant time RSA is error prone
    \item One of the reasons why RSA is not recommended
    \item \emph{Blinding} is typically used with RSA as an additional measure
  \end{itemize}
\end{frame}

\begin{frame}{Constant-timeness verification tools}
  \begin{center}
    \url{https://crocs-muni.github.io/ct-tools/}
  \end{center}
\end{frame}

\begin{frame}{Key takeaways}
  \begin{itemize}[<+(1)->]
    \item Do not roll your own cryptography
    \item Knowing the mathematics is not enough
    \item General purpose math and bignum libraries are not crypto-safe
    \item Even professionals can mess up (e.g. ROCA)
    \item Certification is not foolproof, but is better than nothing
    \item Tools exist to help you validate others' work
  \end{itemize}

  \vspace*{1em}

  \pause
  Additional reading:
  \begin{itemize}
    \item \url{https://timing.attacks.cr.yp.to/programming.html}
    \item Additional references at the bottom
  \end{itemize}
\end{frame}

\end{document}
