\graphicspath{ {../../images/} }
\usetikzlibrary{positioning,calc,external}

\title{ICS0026 Cryptography}
\subtitle{Online voting (a brief introduction)}
\date{\today}
\author{Taaniel Kraavi}
\institute%
{%
  \textit{IT College}\\
  \textit{Tallinn University of Technology}
}

\begin{document}
\begin{frame}
  \titlepage
\end{frame}

\begin{frame}{What's in a name}
  Electronic voting, e-voting
  \begin{itemize}[<+(1)->]
    \item Use of electronic technologies for voting
    \item e.g. electronic voting machine that prints a paper ballot
  \end{itemize}

  \pause
  Online voting, internet voting
  \begin{itemize}[<+(1)->]
    \item Casting a digital vote (e.g. from home)
    \item Some will say electronic voting = online voting (terminology is not standardised)
  \end{itemize}

  \pause
  Estonian internet voting called \enquote{i-voting} in English.
\end{frame}

\begin{frame}{Examples}
  I-voting at a national level:
  \begin{itemize}[<+(1)->]
    \item Estonia (longest running national i-voting system)
    \item Switzerland (digital voting with a postal component)
    \item France (i-voting for expats)
    \item Norway (had trials but discontinued in 2014)
    \item \dots
  \end{itemize}

  \pause
  Tons of online voting companies and technologies exist
  \begin{itemize}[<+(1)->]
    \item various degrees of trust/reliability (not always cryptographic)
    \item cryptographic systems (e.g. IVXV, CHVote, Belenios, Helios, Prêt à Voter, \dots)
  \end{itemize}
\end{frame}

\begin{frame}{Two paradigms}
  We will focus on only two paradigms
  \begin{itemize}[<+(1)->]
    \item Schemes based on homomorphic tallying
    \item Schemes based on cryptographic mixing
  \end{itemize}

  \vspace*{1em}

  \pause
  Why do we need these?
  \begin{itemize}[<+(1)->]
    \item Protect the secret of the ballot (ballot secrecy)
    \item Your choice should remain a secret
    \item A requirement in many democracies
  \end{itemize}

  \vspace*{1em}

  \pause
  In any voting scheme, IND-CPA is required for secrecy!
\end{frame}

\begin{frame}{Homomorphic tally}
  Core idea:
  \begin{itemize}[<+(1)->]
    \item Encrypt \enquote{1} if you vote for the candidate, \enquote{0} otherwise
    \item Aggregate encrypted votes (additively homomorphic scheme!)
    \item Decrypt the aggregated result
  \end{itemize}

  \vspace*{1em}

  \pause
  Sounds good, so what are the caveats?
  \begin{itemize}[<+(1)->]
    \item No arbitrary ballot structure (no write-ins)
    \item Must encrypt one vote for each candidate\textsuperscript{*}
    \item Not practical for large numbers of candidates
  \end{itemize}
\end{frame}

\begin{frame}{Casting a homomorphic ballot}
  Zero-knowledge proof of vote validity:
  \begin{itemize}[<+(1)->]
    \item Prove that you gave only 1 vote for a candidate (not 5 or -1)
    \item Prove that you did not give more than 1 vote in total (typically)
  \end{itemize}

  \pause
  These proofs must be verified before aggregation!

  \vspace*{1em}

  \pause
  Some remarks:
  \begin{itemize}[<+(1)->]
    \item Proving whether a vote is 0 or 1 is much easier than proving arbitrary ranges
    \item Proof should be non-interactive (transferability needed for auditability)
  \end{itemize}
\end{frame}

\begin{frame}{Aggregation and tallying}
  What should an empty ballot box look like?
  \begin{itemize}[<+(1)->]
    \item Either it is really empty\dots
    \item \dots{}or it contains a deterministic \enquote{null-vote}
  \end{itemize}

  \pause
  Decryption
  \begin{itemize}[<+(1)->]
    \item Verify ZKRPs
    \item Aggregate all ciphertexts using the homomorphic property
    \item Decrypt the final results
  \end{itemize}

  \pause
  Aggregation could be done on the fly, but for auditing, the tally should later be independently re-aggregated.
\end{frame}

\begin{frame}{Privacy of aggregation}
  By decrypting the aggregated result, we protect voter identities.
  \begin{itemize}[<+(1)->]
    \item Digital ballots must typically be digitally signed
    \item Decrypting ballots independently breaks vote secrecy
  \end{itemize}

  \vspace*{1em}

  \pause
  How can we prevent the organiser from deciphering all individual ballots?
  \begin{itemize}[<+(1)->]
    \item Secret sharing
    \item Distributed secret sharing on cryptographic HSMs (e.g. JavaCards)
    \begin{itemize}
      \item Key never exists in its entirety
      \item Cards should be destroyed after election procedures end
    \end{itemize}
  \end{itemize} 
\end{frame}

\begin{frame}{(N)everlasting privacy}
  What to do with the digital ballots after the election?
  \begin{itemize}[<+(1)->]
    \item Ballots should be destroyed after the election
    \item In a post-quantum world, the privacy is lost
    \item Organisational measures needed
    \item Always a risk that someone exfiltrates the ballot box
  \end{itemize}
\end{frame}

\begin{frame}{Internet voting in Estonia}
  \begin{itemize}[<+(1)->]
    \item Introduced in 2005
    \item Used in every election since
    \item No known security incidents
    \item 2023: first year where $> 50\%$ of votes were i-votes
    \item Current system called \enquote{IVXV} introduced in 2017
    \item Not a homomorphic scheme: many candidates
  \end{itemize}
\end{frame}

\begin{frame}{Trust in IVXV}
  Controversy after every election but overall trust is actually quite good.
  \begin{itemize}[<+(1)->]
    \item Lots of misconceptions about how the system works
    \item Good official materials are available but few check them out
    \item Anyone can register to observe/audit some of the operations
    \item Some checks only available to professional auditors due to PII
  \end{itemize}
\end{frame}

\begin{frame}{Double envelope postal voting scheme}
  Most i-voting systems are based on the double envelope voting scheme:
  \begin{enumerate}[<+(1)->]
    \item Get two envelopes + a ballot paper in the mail
    \begin{itemize}
      \item One envelope has printed voter identifiers
      \item One envelope is blank
    \end{itemize}
    \item Fill the ballot paper, include it in the unmarked envelope and seal it
    \begin{itemize}
      \item Protects confidentiality
    \end{itemize}
    \item Put the unmarked envelope in the marked envelope, seal it, and sign it
    \begin{itemize}
      \item Provides identification
      \item Signature is supposed to provide some form of authentication (dubious)
    \end{itemize}
  \end{enumerate}

  \pause
  When the envelopes are received, the outer envelope is discarded after verification.
\end{frame}

\begin{frame}{Double envelope in IVXV}
  \begin{enumerate}[<+(1)->]
    \item Make your choice with the voting application
    \item Voting application encrypts the vote (inner envelope)
    \begin{itemize}
      \item A public key is specific for every election
      \item The keypair is generated before the election in a public ceremony
    \end{itemize}
    \item Voting application enables you to sign the vote (identification + authenticity)
  \end{enumerate}

  \vspace*{1em}

  \pause
  Is removing the digital signature enough?
  \begin{itemize}[<+(1)->]
    \item It depends
    \item For homomorphic aggregation? yes
    \item For individual decryption? no
  \end{itemize}
\end{frame}

\begin{frame}{Digital signatures in IVXV}
  IVXV's security relies on the security of the Estonian PKI and eID.
  \begin{itemize}[<+(1)->]
    \item eID exists: authentication is convenient
    \item eID is used for way more than just i-voting
    \begin{itemize}
      \item People trust it because they use it all the time
      \item Electoral organiser does not control the CA or the signing devices
    \end{itemize}
    \item The key distribution problem is solved
    \begin{itemize}
      \item Keys are presumed to be safe
      \item Do not share your PINs!
    \end{itemize}
  \end{itemize}
\end{frame}

\begin{frame}{Coercion and vote selling}
  Voting from home, coercion and vote selling?
  \begin{itemize}[<+(1)->]
    \item Coercion: forced to vote in a specific way or disclose your vote
    \item Vote selling: rewarded for voting in a specific way
  \end{itemize}

  \vspace*{1em}

  \pause
  When voting in person, you should be alone in a voting booth.
  \begin{itemize}[<+(1)->]
    \item Still possible to prove who you voted for (e.g. hidden wearable camera)
  \end{itemize}

  \vspace*{1em}

  \pause
  When voting form home, risks are higher, e.g. shoulder surfing.
  \begin{itemize}[<+(1)->]
    \item The Swiss say that they do not have this problem\dots{} go figure
  \end{itemize}
\end{frame}

\begin{frame}{On mitigating coercion}
  To prevent this, voters can override their vote in two ways in Estonia:
  \begin{itemize}[<+(1)->]
    \item Voters can vote multiple time online
    \begin{itemize}
      \item Only the last received i-vote counts (superseding previous votes)
      \item If someone watches you vote one time, just vote again later
    \end{itemize}
    \item Voters can still go and vote in person
    \begin{itemize}
      \item The physical vote will override the i-vote
    \end{itemize}
    \item Individual attacks possible, but not scalable
  \end{itemize}

  \pause
  This is also the reason why i-voting ends before physical voting ends.
\end{frame}

\begin{frame}{Defining verifiability}
  Verifiability definitions by Kremer et al.
  \begin{itemize}[<+(1)->]
    \item \emph{Individual verifiability}: a voter should be able to check that his vote belongs to the ballot box.
    \item \emph{Universal verifiability}: anyone should be able to check that the result corresponds to the content of the ballot box.
    \item \emph{Eligibility verifiability}: only eligible voters may vote.
  \end{itemize}

  \pause
  \emph{End-to-end} (E2E) verifiable: individually and universally verifiable.
  \begin{itemize}
    \item Not the only E2E definition, but it is the one used in IVXV.
  \end{itemize}

  \pause
  Seemingly, E2E verifiability is mutually exclusive with coercion resistance.
\end{frame}

\begin{frame}{Election keypair}
  Before every election, and election specific keypair is generated.
  \begin{itemize}[<+(1)->]
    \item ElGamal cryptosystem
    \item The \emph{key application} is used to generate the key
    \begin{itemize}
      \item Also used to decrypt votes later
    \end{itemize}
    \item Key is generated in a threshold manner (Shamir)
    \begin{itemize}
      \item The key application is the trusted dealer
      \item Approach with reconstruction needed for performance
      \item Currently, $n = 9$, $t = 5$
    \end{itemize}
  \end{itemize}
\end{frame}

\begin{frame}{Election keyholders}
  The key-shares are written to JavaCards.
  \begin{itemize}[<+(1)->]
    \item Members of the State's Electoral Office and of the National Electoral Committee
    \item Shares are also written on backup cards
    \item Cards are sealed with a tamper evident tape and signed by members present
    \item Backups are stored in a safe by the SEO
  \end{itemize}

  \vspace*{1em}

  \pause
  You can register as an observer to be there (or watch a stream) when this is done
\end{frame}

\begin{frame}{Parties involved}
  Main parties:
  \begin{itemize}[<+(1)->]
    \item The voter (with the voting client)
    \item The collector
    \begin{itemize}
      \item Receives ballots form voters
      \item Provides the configuration to the voting client
    \end{itemize}
    \item The organiser
    \begin{itemize}
      \item Role of the processor
      \item Role of the tallying party
    \end{itemize}
  \end{itemize}

  \pause
  Additional \enquote{tools}:
  \begin{itemize}[<+(1)->]
    \item Verification application
    \item Registration service 
  \end{itemize}
\end{frame}

\begin{frame}{Main processes}
  Drawn and explained on the whiteboard.

  \vspace*{1em}

  \pause
  For students only reading slides but not watching lectures:
  \begin{itemize}[<+(1)->]
    \item This is on the exam
    \item More generally, you are either very good at self study\dots
    \item \dots{}or you will have a big problem in 2 weeks.
  \end{itemize}
\end{frame}

\begin{frame}{Conclusion}
  \begin{itemize}[<+(1)->]
    \item Voting is complicated (not just online voting)
    \item Highly regulated and requirements vary
    \item Definitions vary
    \item Many trade-offs (e.g. privacy vs transparency)
    \item Trust is hard
    \item We barely scratched the surface of online voting
    \begin{itemize}
      \item Different authentication tech
      \item Different verifiability tech (bulletin boards, \dots)
      \item More esoteric schemes (e.g. blockchain voting)
    \end{itemize}
  \end{itemize}
\end{frame}

\end{document}
