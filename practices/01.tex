\documentclass{practice}

\begin{document}
\textbf{Task 1 --- Benchmarking}

Computers are fast, but how fast are they exactly?

Write a C program iterating over the whole space of $32$-bit unsigned integers, and measure the time needed by your CPU to complete this.

How long will it take to iterate over the whole space of $33$-bit unsigned integers?
Confirm your calculation by benchmarking this loop as well.

Finally, calculate how long would it take your CPU to iterate over the whole $64$-bit space.
This should provide some insight into the \textit{intractability} of some computational problems, i.e. the \textit{infeasibility} of solving some tasks in practice due to limited computing power (or some other resource).

Try adding some multiplications or even exponentiations into your $32$-bit loop and benchmark the duration now.

There are of course techniques to speed up computations, e.g. multithreading.
Still, even such techniques are limited.
For example, what about an algorithm that must be executed sequentially?

\textbf{Task 2.a --- a--z shift cipher}

Implement a shift cipher (encryption and decryption functions) for characters a--z in python3.
That is, messages should be all lowercase and contain only letters of the English alphabet, e.g. \texttt{helloworld}.
The ciphertext should likely only contain characters a--z.

Recall that the shift cipher is a cipher where every character of the plaintext alphabet is shifted an equal amount of positions in this alphabet.
This shift is wrap-around, i.e. if you reach the end of the alphabet, you start again from the beginning.

Tip:
recall that in ASCII encoding, consecutive characters of the English alphabet are mapped to consecutive integers.
In python3, you can get the integer value of an ASCII character using the \texttt{ord} function\footnote{\url{https://docs.python.org/3/library/functions.html\#ord}}.

\textbf{Task 2.b --- Frequency analysis}

For both the plaintext and the ciphertext of the a--z shift cipher, plot the letter frequencies on a graph, e.g. with \texttt{matplotlib}\footnote{\url{https://matplotlib.org}}.
Clearly, the shift cipher does not provide diffusion.

\textbf{Task 3 --- Binary shift cipher}

Because it is inconvenient to read long messages without spaces or special characters, and we may want support for more encodings, implement the shift cipher on bytes.
That is, plaintexts of sequence of bytes, e.g. obscure unicode characters, should be able to be encrypted.

Print the ciphertext in hexadecimal encoding.

\textbf{Task 4 --- Substitution cipher}

Implement now the substitution cipher in python3.

The substitution cipher is the cipher where each character of the plaintext alphabet is mapped to some new alphabet character.
Note that the alphabet needs not correspond to a common language, e.g. English.
You could very well map the characters to some other symbols, e.g. emojis.

\textbf{Note}

For tasks 2--4 you may also use some language other than python3.
However, if you do not feel comfortable with python, you should stick with it since it will be required in homework tasks.
\end{document}
