\usetheme[numbering=fraction]{metropolis}

\metroset{block=fill}

\usepackage[dvipsnames]{xcolor}

\usepackage{booktabs}

\usepackage{fancyvrb}
\fvset{listparameters=\setlength{\topsep}{0pt}\setlength{\partopsep}{0pt}}

\usepackage{csquotes}
\setquotestyle{british}

\usepackage{graphicx}
\graphicspath{ {../../images/} }

\usepackage{pgfplots}
\usetikzlibrary{positioning,calc,external}
%\tikzexternalize[prefix=figures/] 

\usepackage{crysymb}

\renewcommand*{\arraystretch}{1.2}

\usepackage{soul}
\usepackage[en-GB]{datetime2}
\DTMlangsetup[en-GB]{ord=omit}

\usetikzlibrary{positioning,calc}
\graphicspath{ {../../images/} }

\title{ICS0026 Cryptography}
\subtitle{Proof systems \& zero-knowledge}
\date{\DTMdate{2024-04-16}}
\author{Taaniel Kraavi}
\institute%
{%
  \textit{IT College}\\
  \textit{Tallinn University of Technology}
}

\begin{document}
\begin{frame}
  \titlepage
\end{frame}

\begin{frame}{Disclaimer}
  \pause
  This lecture introduces some informal notions.

  \pause
  The goal is to
  \begin{itemize}[<+(1)->]
    \item Introduce the concept of zero-knowledge
    \item Develop some cryptographic intuition
    \item Intuitively reason about about protocols
  \end{itemize}

  \vfill

  \pause
  We will not cover any of these topics formally in this course.
\end{frame}

\begin{frame}{Proving things}
  What is a proof?
  \begin{itemize}[<+(1)->]
    \item Proofs in the mathematical sense (i.e. deterministic)
    \begin{itemize}
      \item Deductive argument for some statement
      \item Combination of axioms, rules of inference, and proven statements
    \end{itemize}
    \item Probabilistic \enquote{proofs}: a more relaxed notion
    \begin{itemize}
      \item Approaches of the type \enquote{Alice convinces Bob}
      \item Proof systems in computational complexity theory
    \end{itemize}
  \end{itemize}
\end{frame}

\begin{frame}{(Interactive) proof systems}
  An (interactive) proof system is comprised of two algorithms:
  \begin{itemize}[<+(1)->]
    \item An unbounded prover $\PPP$
    \begin{itemize}
      \item Potentially malicious
    \end{itemize}
    \item A PPT verifier $\VVV$
    \begin{itemize}
      \item Assumed to be honest
    \end{itemize}
  \end{itemize}

  \vspace*{1em}
  
  \pause
  If the prover is also PPT, we use the term \emph{argument} instead of \enquote{proof}.
  \begin{itemize}[<+(1)->]
    \item Often the two terms are used interchangeably (determine from context)
  \end{itemize}

  \vspace*{1em}

  \pause
  The prover and verifier exchange messages.
  \begin{itemize}[<+(1)->]
    \item Messages are exchanged until the verifier makes a decision
    \item Exchanged messages form the \emph{transcript}
  \end{itemize}
\end{frame}

\begin{frame}{Statements and witnesses}
  \pause
  The goal of a proof is to prove or disprove the veracity of a \emph{statement}.
  \begin{itemize}[<+(1)->]
    \item W.l.o.g. we only consider proving the truth
    \item The easiest way to prove something is to provide a \emph{witness} for the claim
  \end{itemize}
\end{frame}

\begin{frame}{Statements and witnesses}
  A bit more formally:
  \begin{itemize}[<+(1)->]
    \item Let $\Rel$ be a decidable binary relation
    \item We call $w$ a witness for a statement $u$ if $(u, w) \in \Rel$
    \item The language
    \[
      \LLL = \{u \mid \exists w : (u, w) \in \Rel\}
    \]
    is the set of statements $u$ that have a witness $w$ in the relation $\Rel$.
  \end{itemize}

  \vfill

  \pause
  To prove the truth of a statement $u$, the prover must show that it belongs to $\LLL$.
  \begin{itemize}[<+(1)->]
    \item The trivial approach is to reveal $w$, but this may not always be possible
  \end{itemize}
\end{frame}

\begin{frame}{Proof system properties}
  \pause
  \emph{Completeness}
  \begin{itemize}[<+(1)->]
    \item An honest verifier should always accept a proof by an honest prover
  \end{itemize}

  \vspace*{1em}

  \pause
  \emph{Soundness}
  \begin{itemize}[<+(1)->]
    \item No prover should be able to produce an accepting proof for an invalid statement, except with some small probability
    \item This probability is called the \emph{soundness error}
  \end{itemize}
\end{frame}

\begin{frame}{Zero-knowledge proof}
  A \emph{zero-knowledge proof} should convince a verifier that a statement is true, without revealing any information beyond the statement's truth.

  \vspace*{1em}

  \pause
  Classical introductory examples:
  \begin{itemize}[<+(1)->]
    \item Ali Baba's cave
    \begin{itemize}
      \item Soundness by repetition
      \item Interactive \& non-transferable
      \item How to make it transferable and what issues arise?
    \end{itemize}
    \item Where's Waldo
    \begin{itemize}
      \item Not perfectly zero-knowledge
    \end{itemize}
  \end{itemize}
\end{frame}

\begin{frame}{Properties of ZKPs}
  A zero-knowledge proof must be:
  \begin{itemize}[<+(1)->]
    \item Complete
    \item Sound
    \item \emph{Zero-knowledge}
    \begin{itemize}
      \item No verifier must learn anything other than the truth of what is being proven
      \item The verifier needs not be honest
      \item The non-honesty is an important subtlety
    \end{itemize}
  \end{itemize}

  \pause
  Completeness and soundness follow from the requirements of a proof system.
\end{frame}

\begin{frame}{ZK proof of knowledge}
  A ZKP does not require a prover to know a witness for the statement.
  \begin{itemize}[<+(1)->]
    \item In most cases ZKPs do imply/require knowledge of a witness
    \item A ZKP is then called a zero-knowledge \emph{proof of knowledge}
  \end{itemize}

  \vspace*{1em}

  \pause
  How to model that a prover must know a witness to create an accepting proof?
\end{frame}

\begin{frame}{Knowledge-soundness}
  The notion of \emph{knowledge-soundness} implies that a prover must know a witness.
  \begin{itemize}[<+(1)->]
    \item We show the existence of an efficient extractor that can recover the witness except with some small probability
  \end{itemize}

  \vspace*{1em}

  \pause
  Knowledge-soundness can be shown via \emph{special soundness}
  \begin{itemize}[<+(1)->]
    \item Often used for $\Sigma$-protocols
    \item We show that a witness can be extracted from two accepting transcripts with identical initial commitments, but differing challenges
  \end{itemize}
\end{frame}

\begin{frame}{Schnorr protocol}
  Schnorr's protocol allows proving knowledge of a discrete logarithm.
  \pause
  \begin{center}
    \begin{tikzpicture}
      \node (alice) at (0,0) {\includegraphics[height=80px]{alice}};
      \node (bob)   at (6,0) {\includegraphics[height=80px]{bob}};

      \node(sk) at (alice.north) [yshift=10px] {$x \in \ZZ_q$};
      \node(pk) at (bob.north)   [yshift=10px] {$h = g^x$};

      \node (n1a) at (alice.east) [yshift=18px] {};
      \node (n1b) at (bob.west)   [yshift=18px] {};
      \node (n2a) at (alice.east) {};
      \node (n2b) at (bob.west)   {};
      \node (n3a) at (alice.east) [yshift=-18px] {};
      \node (n3b) at (bob.west)   [yshift=-18px] {};

      \node at (alice.west) [left,yshift=20px]  {$r \getsu \ZZ_q$};
      \node at (bob.east)   [right,yshift=20px] {$c \getsu \ZZ_q$};

      \draw[->,thick] (n1a) -- (n1b) node[above,midway] {$u = g^r$};
      \draw[<-,thick] (n2a) -- (n2b) node[above,midway] {$c$};
      \draw[->,thick] (n3a) -- (n3b) node[above,midway] {$z = r + cx$};

      \node at (bob.south) [yshift=-10px] {$g^z = g^r g^{cx} \iseq u h^c$};
    \end{tikzpicture}
  \end{center}

  The group $\GG = \langle g\rangle$ must be a DL-group with a prime cardinality $q$.
\end{frame}

\begin{frame}{Schnorr protocol analysis}
  The analysis is done in class.
\end{frame}

\begin{frame}{Transferability}
  Interactive ZKPoKs are not \emph{transferable}
  \begin{itemize}
    \item A verifier not involved in the interaction is not convinced given just the transcript
    \item This follows from the potential non-randomness of challenges
    \item How can we fix the challenge for transferability?
  \end{itemize}

  \vspace*{1em}

  \pause
  If all messages sent from the verifier to the prover are chosen uniformly at random and independently of the prover's messages, a protocol is called \emph{public coin}.
\end{frame}

\begin{frame}{Fiat-Shamir heuristic}
  The Fiat-Shamir transform can be used to transform an interactive public coin protocol into a non-interactive one.
  \begin{itemize}[<+(1)->]
    \item The protocol is then provably secure only in the ROM
    \item Compute the challenge as a random hash function on all previous messages and the statement
    \item FS can be tricky to use: plenty of vulnerabilities have stemmed from not including the statement as an input to the hash function
  \end{itemize}

  \pause
  In practice, the random hash function is instantiated with a strong cryptographic hash function.
  \begin{itemize}[<+(1)->]
    \item Security becomes heuristic
  \end{itemize}
\end{frame}

\end{document}
