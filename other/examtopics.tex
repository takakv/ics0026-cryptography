\documentclass[usegeometry,parskip=half]{scrartcl}

\usepackage[
  a4paper,
  top=25mm,
  bottom=25mm,
  left=30mm,
  right=30mm,
  footskip=10mm
]{geometry}

\usepackage[hidelinks]{hyperref}

\usepackage{csquotes}
\setquotestyle{british}

\usepackage{microtype}
\usepackage{fontspec}
\usepackage{unicode-math}

\setmainfont{Stix Two Text}
\setsansfont{TeX Gyre Heros}
\setmathfont{Stix Two Math} 

\begin{document}

\begin{center}
  \textbf{Exam study guide --- Spring 2024}

  ICS0026 Cryptography

  ver 0.1
\end{center}

Unless stated otherwise, knowing something also means being able to explain it.
If you only need to explain something, you do not need to necessarily recall the thing itself.
For example, you should be able to list the extended CIA triad, on top of simply being able to answer what confidentiality is (knowing \& explaining).
Conversely, you should know what RSA-OAEP is, but you do not need to recall the term yourself (explaining only).

You should be able to\dots{}
\begin{itemize}
  \item \dots{}know the extended CIA triad and be able to explain for each element how cryptography relates to it.
  \item \dots{}know Kerckhoff's principle.
  \item \dots{}distinguish between provable and heuristic security and explain some of their pros and cons.
  \item \dots{}know what confusion and diffusion are and be able to explain them.
  \item \dots{}explain the differences between conditional and unconditional security.
  \item \dots{}explain the differences between a bounded and an unbounded adversary.
  \item \dots{}know the requirements for a One Time Pad.
  \item \dots{}explain what perfect secrecy is and what disadvantages are unavoidable for schemes with perfect secrecy.
  \item \dots{}know the 4 main types of attacks against encryption schemes.
  \item \dots{}explain what is \enquote{true} randomness.
  \item \dots{}explain the importance of high quality randomness in cryptography.
  \item \dots{}know the differences between a TRNG and a PRNG.
  \item \dots{}know explain the differences between deterministic and non-deterministic algorithms.
  \item \dots{}know the basic requirements for a symmetric cryptosystem.
  \item \dots{}know and be able to explain IND-CPA.
  \item \dots{}explain at a high level how stream ciphers work.
  \item \dots{}know that the ChaCha20 stream cipher exists.
  \item \dots{}know the difference between stream and block ciphers.
  \item \dots{}know what modes of operations are and to give two examples.
  \item \dots{}know that the AES block cipher exists.
  \item \dots{}explain why the ECB mode is insecure (Tux!).
  \item \dots{}explain at a high level how it is possible to construct a stream cipher out of a block cipher.
  \item \dots{}know the role of an initialisation vector (IV).
  \item \dots{}explain why authenticated encryption is necessary.
  \item \dots{}know how typically the security level of symmetric ciphers is determined.
  \item \dots{}explain why in disk encryption the password can be changed without having to re-encrypt the entire disk.
  \item \dots{}explain the importance of randomised encryption in public key cryptosystems.
  \item \dots{}know that RSA exists and explain which security assumption is needed for its security (RSA trapdoor).
  \item \dots{}know the discrete logarithm problem (DLP).
  \item \dots{}explain why getting a number in a range with modulo can be insecure, and how to overcome the issue.
  \item \dots{}explain why the two RSA primes must be kept secret.
  \item \dots{}explain why RSA-OAEP should be used over textbook RSA or PKCS\#1 v1.5 padding.
  \item \dots{}know the recommended minimum key-length for 128 bits of security for public key schemes in finite fields.
  \item \dots{}know what a safe prime is.
  \item \dots{}explain why hybrid encryption is useful.
  \item \dots{}explain the Diffie Hellman key-exchange and why it is secure (does not need to be a mathematical explanation).
  \item \dots{}know what ASN.1, DER and PEM are.
  \item \dots{}know what a homomorphic cryptosystem is.
  \item \dots{}explain the difference between a partially and fully homomorphic cryptosystem.
  \item \dots{}explain the re-randomisation property of ElGamal.
  \item \dots{}know the recommended minimum key-length for 128 bits of security for public key schemes on elliptic curves.
  \item \dots{}explain the importance of \enquote{nothing up my sleeve} numbers.
  \item \dots{}explain the importance of knowing how EC parameters have been generated.
  \item \dots{}give one example of a secure elliptic curve.
  \item \dots{}know what NIST, ANSSI, BSI, \dots are (no need to know the acronyms stand for).
  \item \dots{}explain the importance of a KDF.
  \item \dots{}explain what hash functions are.
  \item \dots{}know the requirements for a secure cryptographic hash function.
  \item \dots{}explain collision resistance, first pre-image resistance, and second pre-image resistance.
  \item \dots{}given a target security goal, explain which hash function property is needed to achieve it.
  \item \dots{}give three examples of the utility of cryptographic hash functions.
  \item \dots()know how typically the security level for a hash function is determined.
  \item \dots{}explain why the drunken bishops algorithm is useful.
  \item \dots{}explain the avalanche effect.
  \item \dots{}explain the difference between hashing and encryption.
  \item \dots{}know MD5, SHA1, SHA2, SHA3 and explain which of them are safe to use.
  \item \dots{}explain the usefulness of the SHAKE functions.
  \item \dots{}explain what a length extension attack is.
  \item \dots{}explain what a hash chain is.
  \item \dots{}explain why a cryptographic hash function is not good for password storage.
  \item \dots{}explain what salting is.
  \item \dots{}explain what peppering is.
  \item \dots{}know a secure function for password hashing.
  \item \dots{}explain what a message authentication code is, and why MACs are useful.
  \item \dots{}explain why a dedicated MAC should be used over a regular cryptographic hash function.
  \item \dots{}explain why we need digital signatures.
  \item \dots{}explain the difference between a digital signature and a MAC.
  \item \dots{}explain why MACs do not provide non-repudiation.
  \item \dots{}explain why digital signatures are \emph{transferable} (in the sense that they can be verified by third parties).
  \item \dots{}explain why in many cases, digital signatures alone are not \enquote{enough}.
  \item \dots{}explain the difference between existential forgery and universal forgery (you will get a hint if you do not remember the terms).
  \item \dots{}explain the common approach for signing some data.
  \item \dots{}give two examples of secure signature schemes.
  \item \dots{}explain why nondeterminism is not necessarily a requirement for a signature scheme.
  \item \dots{}know what the DSS (digital signature standard) is.
  \item \dots{}describe an approach for long term validity of digital signatures.
  \item \dots{}explain the issue with long term authenticity.
  \item \dots{}explain the issue with long term confidentiality.
  \item \dots{}know the two constructions on which post-quantum signature schemes are based on.
  \item \dots{}explain at a high level what a group/ring signature scheme is.
  \item \dots{}explain at a high level what a blind signature is.
  \item \dots{}describe some approach to how digital trust can be achieved.
  \item \dots{}explain what the public key infrastructure is.
  \item \dots{}explain what the web of trust is.
  \item \dots{}explain what a digital certificate is and why it is needed.
  \item \dots{}describe what should typically be part of a digital certificate.
  \item \dots{}describe the importance and role of a certification authority.
  \item \dots{}know what X.509 is.
  \item \dots{}explain how certificate revocation lists work.
  \item \dots{}explain how OCSP works.
  \item \dots{}argue for/against CRLs and OCSP.
  \item \dots{}explain what OCSP stapling is and to argue why it is a good thing.
  \item \dots{}know what eIDAS is (no need to know what the acronym stands for).
  \item \dots{}know what an eID is.
  \item \dots{}know the signature levels of eIDAS.
  \item \dots{}explain the general differences between the three signature levels.
  \item \dots{}explain why qualified signature creation devices are important.
  \item \dots{}explain the importance of trusted timestamping.
  \item \dots{}know the difference between a digital signature and an electronic signature.
  \item \dots{}know what ASiC-E is (what it is used for).
  \item \dots{}know what is SSL/TLS.
  \item \dots{}give a general overview of the TLS handshake.
  \item \dots{}give an example of why TLS1.3 is a good upgrade over TLS1.2.
  \item \dots{}know that you should not use TLS less than v1.2.
  \item \dots{}explain the need for perfect forward secrecy.
  \item \dots{}explain how root certificates get their trust.
  \item \dots{}know what Let's Encrypt is.
  \item \dots{}explain what a secure cryptoprocessor is.
  \item \dots{}explain what a TPM is.
  \item \dots{}explain what a HSM is.
  \item \dots{}explain what a SmartCard/JavaCard is.
  \item \dots{}explain what side channels are.
  \item \dots{}give two examples of side channels (categories).
  \item \dots{}give two examples on how to mitigate side channels.
  \item \dots{}explain the role of commitment schemes.
  \item \dots{}explain the two main requirements that commitment schemes should satisfy.
  \item \dots{}explain why a commitment scheme cannot be perfectly hiding and perfectly binding.
  \item \dots{}describe the secure coin flip over the telephone.
  \item \dots{}explain how can a cryptosystem be turned into a commitment scheme.
  \item \dots{}explain the importance of having verifiable parameter generation for commitment schemes.
  \item \dots{}explain the difference between a proof of knowledge and a proof of possession.
  \item \dots{}explain what a sigma protocol is.
  \item \dots{}explain what a proof system is.
  \item \dots{}explain the difference between an interactive and non-interactive proof and what ramifications this has.
  \item \dots{}explain the principle of zero knowledge.
  \item \dots{}explain the Fiat-Shamir heuristic.
  \item \dots{}explain what secret sharing is.
  \item \dots{}know the meaning of $(t, n)$-threshold schemes and to reason about different $t$ and $n$ values.
  \item \dots{}know what Shamir's scheme is and what technique it relies upon.
  \item \dots{}explain what secure multiparty computation is.
  \item \dots{}give examples of the use of MPC.
  \item \dots{}know what a beaver triple is used for.
\end{itemize}
\end{document}
