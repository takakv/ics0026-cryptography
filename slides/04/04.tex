\graphicspath{ {../../images/} }
\usetikzlibrary{external}

\title{ICS0026 Cryptography}
\subtitle{Public-key cryptography \& RSA}
\date{\today}
\author{Taaniel Kraavi}
\institute%
{%
  \textit{IT College}\\
  \textit{Tallinn University of Technology}
}

\begin{document}
\begin{frame}
  \titlepage
\end{frame}

\begin{frame}{Public-key cryptosystem}
  A \emph{public-key encryption scheme} is a triple of algorithms:
  \begin{itemize}[<+(1)->]
    \pause\item $\GEN$ is a randomised \emph{key generation algorithm}: $(\PK,\SK)\gets\GEN$
    \pause\item $\ENC$ is a (possibly randomised) \emph{encryption algorithm}: $c\gets\ENC_\PK(m)$
    \pause\item $\DEC$ is a \emph{decryption algorithm}: $m\gets\DEC_\SK(c)$.
  \end{itemize}

  \pause
  There are two keys:
  \begin{itemize}[<+(1)->]
    \item $\SK$ is the \emph{secret key}, also called \emph{private key}
    \item $\PK$ is the \emph{public key}
    \item $\PK$ is derived from the $\SK$
    \item It should be \emph{infeasible} to derive $\SK$ from the $\PK$
  \end{itemize}

  \pause
  The cryptosystem is \emph{functional} if $\DEC_\SK(\ENC_\PK(m))=m$.
\end{frame}

\begin{frame}{Public-key cryptosystem}
  \begin{center}
    \begin{tikzpicture}
      \node (alice) at (0,0)      {\includegraphics[height=80px]{alice}};
      \node (bob)   at (8, 0)     {\includegraphics[height=80px]{bob}};
      \node (eve)   at (4, -2.1)  {\includegraphics[height=80px]{eve}};
  
      \draw[->,thick] (alice.east) -- (bob.west);
      
      \node[anchor=east] (gen) at (11, 2) {$(\PK,\SK)\gets\GEN$};
      \draw[<-,dashed,thick] (alice.north) |- (gen.west);

      \node (el) at (eve.north) [xshift=-5px] {};
      \node (er) at (eve.north) [xshift=5px] {};

      \path let \p1 = (el) in node (circ1) at (\x1,2) {};
      \path let \p1 = (er) in node (circ2) at (\x1,0) {};

      \node[anchor=west] (m) at (-3, 0.25) {$m\gets\MMM$};
      \node[anchor=west] (enc) at (-3, -0.25) {$c\gets\ENC_\PK(m)$};
      \node[anchor=east] (dec) at (11, 0) {$m\gets\DEC_\SK(c)$};
      
      \node (c) at (circ2.north) [yshift=5px] {$c$};

      \draw[->,dashed,thick] (circ1) -- (el);
      \draw[->,thick] (circ2.center) -- (er);

      \node[anchor=center,draw,circle,fill=white,thick,minimum size = 2px, inner sep=0pt] at (circ1) {};
      \node[anchor=center,draw,circle,fill=white,thick,minimum size = 2px, inner sep=0pt] at (circ2) {};
  
      \node (ska) at (circ1.north) [yshift=5px] {$\PK$};
    \end{tikzpicture}
  \end{center}

  \pause
  The public key $\PK$ should be known to everyone.
\end{frame}

\begin{frame}{Security of PK systems}
  \begin{itemize}[<+(1)->]
    \item \enquote{Provable} rather than heuristic/best-effort
    \item Relies on the \emph{hardness} of problems (mathematical and/or computational)
    \begin{itemize}
      \item Integer factorisation
      \item Discrete logarithm problem (DLP)
      \item Lattice problems
    \end{itemize}
    \item Public key derived from secret key: mathematical link
    \item Anyone can encrypt with the public key
    \begin{itemize}
      \item No need for \emph{encryption oracles}
      \item CPA setting becomes trivial
      \item \emph{Key exchange} no longer a security problem\textsuperscript{*}
    \end{itemize}
    \item Easy to overlook timing side-channels in implementations!
  \end{itemize}
\end{frame}

\begin{frame}{Modular arithmetic}
  \begin{itemize}[<+->]
    \item \enquote{Secures} most of classical public-key cryptography
    \begin{itemize}
      \item Integer factorisation
      \item Discrete logarithm problem (DLP)
    \end{itemize}
    \item Intractability of some operations given numbers large enough
    \begin{itemize}
      \item Intractable for classical computers
      \item Solvable (eventually) by quantum computers (Shor's algorithm)
      \item 48-bit factorisation claimed by Yan et al. (Dec. 2022)
    \end{itemize}
    \item \enquote{Clock arithmetic}
  \end{itemize}
\end{frame}

\begin{frame}{Terminology refresher}
  Two integers are \alert{congruent} modulo $n$ if
  \[
    a \equiv b \pmod{n},
  \]
  \pause
  that is, if $n \mid a - b$ (read as: $n$ divides $a-b$).

  \vspace*{1em}

  \pause
  You will often see the notation
  \[
    a \bmod n = b
  \]
  due to its familiarity among programmers:
  \begin{center}
    \texttt{b = a \% n}.
  \end{center}
\end{frame}

\begin{frame}{Terminology refresher}
  In both
  \begin{itemize}
    \item $a \equiv b \pmod{n}$ 
    \item $a \bmod n = b$,
  \end{itemize}
  $n$ is called the \alert{modulus}.

  \pause
  Some terminology you might encounter:
  \begin{itemize}[<+(1)->]
    \item $\ZZ_n$: \emph{ring} of integers modulo $n$
    \begin{itemize}
      \item Essentially just the set of values $\{0, 1, \dots, n-1\}$ (a bit simplified)
    \end{itemize}
    \item $\ZZ_n^*$: integers modulo $n$ with modular multiplicative inverses
    \begin{itemize}
      \item The \emph{multiplicative group} of integers modulo $n$
      \item Integers for which there exists $a^{-1}$ such that $a\cdot a^{-1} \equiv 1 \pmod{n}$
    \end{itemize}
  \end{itemize}
\end{frame}

\begin{frame}{Modulo bias}
  How should we generate $n$-bit numbers in a range?

  \pause
  Naive approach:
  \begin{itemize}[<+(1)->]
    \item Generate a random $n$-bit number
    \item Take the number modulo the upper limit
  \end{itemize}

  \pause
  This approach introduces bias!
\end{frame}

\begin{frame}{Modulo bias}
  \begin{figure}
    \includegraphics[height=200px]{bias}
  \end{figure}
\end{frame}

\begin{frame}{Modulo bias}
  \pause
  We must use \alert{rejection sampling} instead!

  \vspace*{1em}

  \pause
  Example from the GO standard library:
  \url{https://go.dev/src/crypto/rand/util.go}

  \vspace*{1em}

  \pause
  Other examples (not an endorsement):
  \url{https://www.pcg-random.org/posts/bounded-rands.html}
\end{frame}

\begin{frame}{Integer factorisation}
  \pause
  \begin{block}{Fundamental theorem of arithmetic}
    Every integer greater than $1$ has a unique prime factorisation.
  \end{block}

  \pause
  Testing for primality is efficient (polynomial time)
  \pause
  \begin{itemize}
    \item For large numbers, tests are typically probabilistic
  \end{itemize}
  
  \pause
  Efficient factorisation is unsolved on classical computers
  \begin{itemize}[<+(1)->]
    \item No polynomial time algorithm
    \item Shor's algorithm for quantum computers
    \item Special purpose algorithms exist
  \end{itemize}
\end{frame}

\begin{frame}{RSA Numbers}
  \pause
  \begin{block}{Semiprime}
    A number with exactly two prime factors.
  \end{block}
  
  \begin{itemize}[<+(1)->]
    \item RSA Factoring Challenge (1991)
    \item RSA numbers: published semiprimes
    \item Discontinued in 2007
  \end{itemize}

  \pause
  Largest RSA number factored to date:
  \begin{itemize}[<+(1)->]
    \item RSA-250: 829-bit (Feb. 2020)
    \item $\sim$2700 CPU core-years
    \item 2.1 GHz Intel Xeon Gold 6130 CPU
  \end{itemize}
\end{frame}

\begin{frame}{RSA cryptosystem}
  Public-key cryptosystem by Rivest, Shamir \& Adleman (1977).
  
  \begin{enumerate}[<+(1)->]
    \item Generate two secret large primes $p,q$ (caveats apply)
    \item Compute $n=pq$, and $\varphi(n)=(p-1)(q-1)$
    \item $e \getsu \ZZ_{\varphi(n)}^*$ and set $d = e^{-1} \bmod \varphi(n)$
    \item $\SK=(p,q,e,d)$ and $\PK = (n,e)$
  \end{enumerate}

  \pause
  Variants exist:
  \begin{itemize}[<+(1)->]
    \item Carmichael's totient $\lambda$ over Euler's totient $\varphi$ (e.g. \href{https://csrc.nist.gov/pubs/fips/186-5/final}{FIPS 186-5})
    \item Originally: pick $d$, compute $e$ (optimises decryption)
    \item More caveats: do not implement RSA yourself!
    \item DYOR when choosing an implementation
  \end{itemize}
\end{frame}

\begin{frame}{RSA cryptosystem}
  Encryption:
  \pause
  \begin{itemize}
    \item $c \equiv m^e \pmod{n}$
  \end{itemize}

  \pause
  Decryption:
  \pause
  \begin{itemize}
    \item $c^d \equiv \bigl(m^e\bigr)^d \equiv m \pmod{n}$
  \end{itemize}
\end{frame}

\begin{frame}{Modular exponentiation}
  Naive exponentiation $m^e \bmod n$ is wasteful.

  \pause
  Modular exponentiation is a technique to improve this computation.

  \pause
  \[
    a^b \bmod c = \Bigl((a \bmod c)^b\Bigr) \bmod c
  \]

  \pause
  Fast modular exponentiation
  \begin{itemize}[<+(1)->]
    \item Exponentiation by squaring
    \item Nice intro on this \href{https://web.archive.org/web/20240303145251/https://dev-notes.eu/2019/12/Fast-Modular-Exponentiation/}{blog}
  \end{itemize}

  \pause
  \begin{block}{Warning!}
    Very vulnerable to timing-based side channels!
  \end{block}
  
\end{frame}

\begin{frame}{Issues with \enquote{textbook} RSA}
  Textbook RSA is:
  \begin{itemize}[<+(1)->]
    \item Deterministic (not IND-CPA)
    \item Vulnerable to small-exponent attacks
    \begin{itemize}
      \item Coppersmith's attack (small public key)
      \item Wiener's attack (small private key)
    \end{itemize}
    \item Malleable (not IND-CCA)
  \end{itemize}

  \pause
  Padding schemes can improve these issues somewhat.

  \pause
  Padding schemes do not protect against fatal mistakes, e.g.
  \begin{itemize}[<+(1)->]
    \item re-using primes with different keys
    \item using weak or small primes (\href{https://cybersec.ee/2017/10/18/rsa-2048-bit-keys-in-estonian-id-cards-issued-after-october-2014-are-factorizable/}{Estonian example})
  \end{itemize}
\end{frame}

\begin{frame}{RSA padding schemes}
  \pause
  PKCS \#1:
  \begin{itemize}[<+(1)->]
    \item \texttt{RSAES-PKCS1-v1\_5} vulnerable!
    \item \texttt{RSAES-OAEP} (PKCS \#1 v2.0)
    \item \href{https://datatracker.ietf.org/doc/html/rfc8017}{RFC 8017} -- PKCS \#1 v2.2
  \end{itemize}

  \pause
  OAEP --- Optimal Asymmetric Encryption Padding
  \begin{itemize}[<+(1)->]
    \item Provides randomness (probabilistic encryption)
    \item Prevents partial decryption
    \item IND-CCA1 (non-adaptive)
    \item Not IND-CCA2 (adaptive)
  \end{itemize}
\end{frame}

\begin{frame}{Security level}
  PK cryptography typically requires larger keys to achieve the same security level as symmetric cryptosystems.

  \pause
  Some shaky explanations:
  \begin{itemize}[<+(1)->]
    \item Brute force is \enquote{easier}
    \item Not all keys in the bit-space are valid
    \item Mathematical problems with various algorithms to attempt to solve them
    \item Relationship between public and secret key
  \end{itemize}
\end{frame}

\begin{frame}{Security level}
  Estimates defined in \href{https://csrc.nist.gov/pubs/sp/800/57/pt1/r5/final}{NIST SP 800-57 REV. 5}

  \pause
  \begin{center}
  \begin{tabular}{|c|c|c|}
    Security Strength & Symmetric & RSA\\
    128 & AES-128 & $k = 3072$\\
    192 & AES-192 & $k = 7680$\\
    256 & AES-256 & $k = 15360$
  \end{tabular}
  \end{center}

  \pause
  The table assumes classical adversaries!

  \pause
  Other recommendations:
  \begin{itemize}
    \item ANSSI: \href{https://cyber.gouv.fr/le-referentiel-general-de-securite-version-20-les-documents}{RGS v2.0 annex B1}
    \item BSI: \href{https://www.bsi.bund.de/EN/Themen/Unternehmen-und-Organisationen/Standards-und-Zertifizierung/Technische-Richtlinien/TR-nach-Thema-sortiert/tr02102/tr02102_node.html}{TR-02102-1}
    \item NCSC (NL): \href{https://english.ncsc.nl/publications/publications/2021/january/19/it-security-guidelines-for-transport-layer-security-2.1}{IT Security Guidelines for
    Transport Layer Security (TLS)\textsuperscript{*}}
    \item \url{https://www.keylength.com/en/compare/}
  \end{itemize}
\end{frame}

\begin{frame}{Hybrid encryption}
  Public-key cryptosystems are slow, but \enquote{solve} the key exchange problem\textsuperscript{*}.

  \pause
  Hybrid cryptosystem:
  \begin{itemize}[<+(1)->]
    \item \emph{key encapsulation mechanism} (KEM)
    \item \emph{data encapsulation scheme}
  \end{itemize}

  \pause
  Key encapsulation mechanisms:
  \begin{itemize}[<+(1)->]
    \item Not simply PK encryption: there are changes
    \item Key derivation functions (KDF) often used
    \item Example: RSA-KEM (\href{https://datatracker.ietf.org/doc/html/rfc5990}{RFC 5990})
  \end{itemize}

  \pause
  We will talk about key exchange algorithms next week.
\end{frame}

\begin{frame}{Signature scheme pitfall}
  \pause
  \begin{block}{Common misconception}
    By \enquote{encrypting} with a secret key and \enquote{decrypting} with a public key, a public-key encryption scheme becomes a signature scheme.
  \end{block}

  \begin{itemize}[<+(1)->]
    \item Under no circumstance shall you do this!
    \item Generally it does not even work (formatting issues)
    \item Different security requirements
    \item Use separate signing keys
    \item Subtleties that you do not even know exist!
  \end{itemize}

  \pause
  \textbf{Use the right tool for the right task. Never assume!}
\end{frame}

\begin{frame}{ASN.1}
  Abstract Syntax Notation One
  \begin{itemize}[<+(1)->]
    \item Many encoding rules
    \item Distinguished Encoding Rules (DER)
    \item DER often used in cryptography: unequivocal
  \end{itemize}

  \pause
  Some tools:
  \begin{itemize}[<+(1)->]
    \item \url{https://asn1.io/asn1playground/}
    \item \url{https://lapo.it/asn1js/}
    \item \texttt{dumpasn1} (CLI)
  \end{itemize}
\end{frame}

\end{document}
